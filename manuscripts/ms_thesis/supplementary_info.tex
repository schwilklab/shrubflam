\documentclass[12pt]{report}
\usepackage{graphicx}

\usepackage{titling}

\pretitle{\begin{center}\Large\bfseries}
\posttitle{\par\end{center}\vspace{2em}}

\preauthor{\begin{center}\large}
\postauthor{\par\end{center}\vspace{1em}}

\title{Canopy traits drive plant flammability: shoot flammability of Texas shrubs}
\author{Azaj Mahmud\textsuperscript{1} and Dylan W. Schwilk, Ph.D.\textsuperscript{2,*}\\
\textsuperscript{1}Department of Biological Sciences, Texas Tech University, Lubbock, Texas, USA\\
\textsuperscript{2,*}Associate Professor, Department of Biological Sciences, Texas Tech University, Lubbock, Texas, USA\\
\vspace{1em}
\textsuperscript{*}Corresponding author: dylan.schwilk@ttu.edu}
\date{}


\begin{document}
\maketitle


\begin{figure}
\centering
\caption{Kendall rank correlation coefficient plot for all the measured morphological traits.}
\includegraphics[width = \textwidth]{../../results/morphological_traits_correlation_plot.pdf}
\end{figure}

\clearpage

\begin{table}[ht]
\centering
\caption{Result of model using Kenward--Roger approximation for the best canopy traits model for temperature integration above 100 $^{\circ}$C. Marginal $R^2$ and conditional $R^2$ are highlighted in bold.}
\begin{tabular}{rllll}
 \hline
 & Predictors & Estimates & CI & p
 \\ 
  \hline
 & (Intercept) & 14051.18 & 9946.05 – 18156.31 & $<$0.001 \\ 
  3 & total dry mass per 70\,cm (g) & 14075.41 & 11404.63 – 16746.19 & $<$0.001 \\ 
  4 & canopy density (g/{$cm^3$}) & 3054.81 & 763.97 – 5345.65 & $<$0.001 \\ 
  5 & Random Effects &  &  &  \\ 
  6 & $\sigma^2$ & 97196044.34 &  &  \\ 
  7 & $\tau_{00 genus}$ & 46961553.69 &  &  \\ 
  8 & ICC & 0.33 &  &  \\ 
  9 & $N_{genus}$ & 14 &  &  \\ 
  10 & Marginal $R^2$ / Conditional $R^2$ & \textbf{0.639} / \textbf{0.756} &  &  \\ 
   \hline
\end{tabular}
\end{table}

\begin{table}[ht]
\centering
\caption{Result of model using Kenward--Roger approximation for the best leaf traits model for temperature integration above 100 $^{\circ}$C. Marginal $R^2$ and conditional $R^2$ are highlighted in bold.}
\begin{tabular}{rllll}
  \hline
 & Predictors & Estimates & CI & p \\ 
  \hline
2 & (Intercept) & 10446.18 & 2785.32 – 18107.04 & $<$0.001 \\ 
  3 & LMA & 4810.53 & 920.53 – 8700.53 & $<$0.001 \\ 
  4 & Random Effects &  &  &  \\ 
  5 & $\sigma^2$ & 206214897.13 &  &  \\ 
  6 & $\tau_{00 genus}$ & 184480723.11 &  &  \\ 
  7 & ICC & 0.47 &  &  \\ 
  8 & $N_{genus}$ & 14 &  &  \\ 
  9 & Marginal $R^2$ / Conditional $R^2$ & \textbf{.056} / \textbf{.502} &  &  \\ 
   \hline
\end{tabular}
\end{table}

\begin{table}
\centering
\caption{Result of model using Kenward--Roger approximation for the best canopy traits model for ignition delay. Marginal $R^2$ and conditional $R^2$ are highlighted in bold.}
\begin{tabular}{rllll}
  \hline
 & Predictors & Estimates & CI & p \\ 
  \hline
2 & (Intercept) & 2.94 & 2.17 – 3.72 & $<$0.001 \\ 
3 & canopy density (g/{$cm^3$}) & 1.12 & 0.56 – 1.68 & $<$0.001 \\ 
  4 & canopy moisture content (\%) & 0.94 & 0.33 – 1.55 & $<$0.001 \\ 
  5 & Random Effects &  &  &  \\ 
  6 & $\sigma^2$ & 7.38 &  &  \\ 
  7 & $\tau_{00 genus}$ & 1.17 &  &  \\ 
  8 & ICC & 0.14 &  &  \\ 
  9 & $N_{genus}$ & 14 &  &  \\ 
  10 & Marginal $R^2$ / Conditional $R^2$ & \textbf{0.201} / \textbf{.311} &  &  \\  
   \hline
\end{tabular}
\end{table}


\begin{table}[ht]
\centering
\caption{Result of model using Kenward--Roger approximation for the best leaf traits model for ignition delay. Marginal $R^2$ and conditional $R^2$ are highlighted in bold.}
\begin{tabular}{rllll}
  \hline
 & Predictors & Estimates & CI & p \\ 
  \hline
2 & (Intercept) & 2.76 & 1.94 – 3.58 & $<$0.001 \\ 
  3 & LMA & 1.02 & 0.39 – 1.65 & $<$0.001 \\ 
  4 & leaf moisture content & 0.86 & 0.22 – 1.49 & $<$0.001 \\ 
  5 & Random Effects &  &  &  \\ 
  6 & $\sigma^2$ & 7.68 &  &  \\ 
  7 & $\tau_{00 genus}$ & 1.40 &  &  \\ 
  8 & ICC & 0.15 &  &  \\ 
  9 & $N_{genus}$ & 14 &  &  \\ 
  10 & Marginal $R^2$ / Conditional $R^2$ & \textbf{.143} / \textbf{.275} &  &  \\ 
   \hline
\end{tabular}
\end{table}

\begin{table}
\centering
\caption{Mixed effect model coefficients and ANOVA table for the best predictors for each type of traits for temperature integration over 100 $^{\circ}$C without three species from \emph{Juniperus} taxon. The linear mixed model was fitted with lmer() function from lme4 package. The car package in R  was used to calculate the estimated degrees of freedom of residuals, F and p-values using the Kenward-Roger approximation. All the independent variables were standardized  as z-score. P values are highlighted in bold.}
\vspace{0.5 cm}
\begin{tabular}{lrrrr}
  \hline
 &  Estimate & F  & Df.res & Pr($>$F) \\ 
  \hline 
  total dry mass (g) & 9549.250 & 40.881  & 85.225 & \textbf{$<$0.001} \\ 
  canopy density (g/{$cm^3$}) & 6582.310 & 0.824  & 80.418 & \textbf{0.367} \\ 
  LMA & 99.640 & 0.016 &  86.503 & \textbf{0.901} \\ 
   \hline
\end{tabular}
\end{table}



\begin{table}
\centering
\caption{Mixed effect model coefficients and ANOVA table for the best predictors for each type of traits for ignition delay without three species from \emph{Juniperus} taxon. The linear mixed model was fitted with lmer() function from lme4 package. The car package in R  was used to calculate the estimated degrees of freedom of residuals, F and p-values using the Kenward-Roger approximation. All the independent variables were standardized  as z-score. P values are highlighted in bold.}
\vspace{0.5 cm}
\begin{tabular}{lrrrr}
  \hline
 &  Estimate & F  & Df.res & Pr($>$F) \\ 
  \hline 
  canopy density (g/{$cm^3$}) & 0.199 &  0.647 & 85.732 & \textbf{0.423} \\ 
  LMA  & 0.238 & 0.965 & 66.079 & \textbf{0.330} \\ 
  leaf moisture content & 0.446 & 4.037 & 80.261  & \textbf{0.048} \\ 
   \hline
\end{tabular}
\end{table}

\begin{table}[ht]
\centering
\caption{The maximum leaf moisture content (\%) and maximum ignition delay in seconds in this experiment with their corresponding species.}
\begin{tabular}{lrr}
  \hline
  \textbf{species name} & \textbf{leaf moisture content} & \textbf{ignition delay} \\ 
  \hline
  \emph{Juniperus pinchotii} & 58.03 & 18.00 \\ 
  \emph{Forestiera pubescens} & 108.05 & 2.00 \\ 
   \hline
\end{tabular}
\end{table}






\end{document}


