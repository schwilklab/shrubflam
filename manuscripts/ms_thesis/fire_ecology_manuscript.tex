\documentclass{bmcart}
%%% Load packages
%\usepackage{amsthm,amsmath}
%\RequirePackage{natbib}
%\RequirePackage[authoryear]{natbib}% uncomment this for author-year bibliography
%\RequirePackage{hyperref}
\usepackage[utf8]{inputenc} %unicode support
%\usepackage[applemac]{inputenc} %applemac support if unicode package fails
%\usepackage[latin1]{inputenc} %UNIX support if unicode package fails
%%%%%%%%%%%%%%%%%%%%%%%%%%%%%%%%%%%%%%%%%%%%%%
%Include any other add-on  packages you need:%
%%%%%%%%%%%%%%%%%%%%%%%%%%%%%%%%%%%%%%%%%%%%%%
\let\bibhang\relax % ttuthes2007 is not easily compatible with natbib

\usepackage{natbib}
\bibpunct{(}{)}{,}{a}{}{,}
\usepackage{amsmath,graphicx}
%\usepackage[utf8]{inputenc}
\usepackage[T1]{fontenc}
\usepackage{lmodern}
\usepackage[labelfont=bf,labelsep=period]{caption}
%\usepackage{multirow}
%\usepackage{textcomp}
\usepackage[colorlinks=true,citecolor=black,linkcolor=black]{hyperref}
\usepackage{lscape} % for making table as landscape
%\usepackage{enumitem}
\usepackage{lineno}
%\usepacakge{blindtext}
\usepackage{setspace}
\usepackage{booktabs}
%\usepackage[top = 1in, left=0.5in,
%right=.5in, bottom=1in]{geometry}
\usepackage{xcolor}
\usepackage{url}
\renewcommand\UrlFont{\color{black}\ttfamily} 

% Format for typesetting R packages
\newcommand{\pkg}[1]{\textsc{#1}} 


%%%%%%%%%%%%%%%%%%%%%%%%%%%%%%%%%%%%%%%%%%%%%%%
%Uncomment if the grad school doesn't like the%
%line under the  running head:                %
%%%%%%%%%%%%%%%%%%%%%%%%%%%%%%%%%%%%%%%%%%%%%%%
%\renewcommand{\headrulewidth}{0pt}




%\def\includegraphic{}
%\def\includegraphics{}



%%% Put your definitions there:
\startlocaldefs
\endlocaldefs


%%% Begin ...
\begin{document}

%%% Start of article front matter
\begin{frontmatter}

\begin{fmbox}
\dochead{Fire Ecology}

%%%%%%%%%%%%%%%%%%%%%%%%%%%%%%%%%%%%%%%%%%%%%%
%%                                          %%
%% Enter the title of your article here     %%
%%                                          %%
%%%%%%%%%%%%%%%%%%%%%%%%%%%%%%%%%%%%%%%%%%%%%%

\title{Canopy traits drive plant flammability: shoot flammability of Texas shrubs}

%%%%%%%%%%%%%%%%%%%%%%%%%%%%%%%%%%%%%%%%%%%%%%
%%                                          %%
%% Enter the authors here                   %%
%%                                          %%
%% Specify information, if available,       %%
%% in the form:                             %%
%%   <key>={<id1>,<id2>}                    %%
%%   <key>=                                 %%
%% Comment or delete the keys which are     %%
%% not used. Repeat \author command as much %%
%% as required.                             %%
%%                                          %%
%%%%%%%%%%%%%%%%%%%%%%%%%%%%%%%%%%%%%%%%%%%%%%

\author[
   addressref={aff1},                   % id's of addresses, e.g. {aff1,aff2}
   email={azaj.mahmud@email.com}        % email address of first author
]{\inits{AM}\fnm{Azaj} \snm{Mahmud}}
\author[
   addressref={aff1,aff2},
   corref={aff2},                       % id of corresponding address
   email={dylan.schwilk@ttu.edu}        % email address of corresponding author
]{\inits{DWS}\fnm{Dylan W} \snm{Schwilk}}


%\author[
   %addressref={aff1},                   % id's of addresses, e.g. {aff1,aff2}
   %corref={aff2},                       % id of corresponding address, if any
   %noteref={n1},                        % id's of article notes, if any
   %email={dylan.schwilk@ttu.edu}   % email address
%]{\inits{AM}\fnm{Azaj} \snm{Mahmud}}
%\author[
   %addressref={aff1,aff2}
%]{\inits{DWS}\fnm{Dylan W} \snm{Schwilk}}


%%%%%%%%%%%%%%%%%%%%%%%%%%%%%%%%%%%%%%%%%%%%%%
%%                                          %%
%% Enter the authors' addresses here        %%
%%                                          %%
%% Repeat \address commands as much as      %%
%% required.                                %%
%%                                          %%
%%%%%%%%%%%%%%%%%%%%%%%%%%%%%%%%%%%%%%%%%%%%%%

\address[id=aff1]{%                           % unique id
  \orgname{Department of Biological Sciences, Texas Tech University}, % university, etc
  \street{Lubbock},                     %
  %\postcode{}                                % post or zip code
  \city{Texas},                              % city
  \cny{USA.}                                    % country
}
\address[id=aff2]{%
  \orgname{Associate Professor,  Department of Biological Sciences, Texas Tech University}, % university, etc
  \street{Lubbock},                     %
  %\postcode{}                                % post or zip code
  \city{Texas},                              % city
  \cny{USA.}                                    % country
}

%%%%%%%%%%%%%%%%%%%%%%%%%%%%%%%%%%%%%%%%%%%%%%
%%                                          %%
%% Enter short notes here                   %%
%%                                          %%
%% Short notes will be after addresses      %%
%% on first page.                           %%
%%                                          %%
%%%%%%%%%%%%%%%%%%%%%%%%%%%%%%%%%%%%%%%%%%%%%%

%\begin{artnotes}
%\note{Sample of title note}     % note to the article
%\note[id=n1]{Equal contributor} % note, connected to author
%\end{artnotes}

\end{fmbox}% comment this for two column layout

%%%%%%%%%%%%%%%%%%%%%%%%%%%%%%%%%%%%%%%%%%%%%%
%%                                          %%
%% The Abstract begins here                 %%
%%                                          %%
%% Please refer to the Instructions for     %%
%% authors on http://www.biomedcentral.com  %%
%% and include the section headings         %%
%% accordingly for your article type.       %%
%%                                          %%
%%%%%%%%%%%%%%%%%%%%%%%%%%%%%%%%%%%%%%%%%%%%%%

\begin{abstractbox}

\begin{abstract} % abstract
\parttitle{Background} %if any
Flammability is a trait of land plants, and an understanding of this trait is important to comprehend the ecological and evolutionary consequences of fire on plant species in fire-prone ecosystems. Numerous plant traits can influence flammability. Although many studies have examined the flammability of individual leaves, fewer have examined how entire canopies behave as fuel because quantifying plant flammability by burning whole plants is expensive. However, a recent study has suggested that burning shoots can predict plant flammability. In this study, we used a new plant flammability device that allows standardized measurements of canopy flammability of portions up to 70\,cm long. We burned 70\,cm branches from at least three samples per species of 16 native shrubs of Texas and measured four canopy traits: total dry mass per 70\,cm, canopy density, leaf: stem (in dry mass basis) and canopy moisture content and four common leaf traits: leaf mass per area (LMA), leaf area per leaflet, leaf length per leaflet and leaf moisture content. We used these data to answer two questions: 1) Are canopy traits more important than leaf traits in shoot flammability? 2) Are heat release and flame spread rate independent axes of flammability in shrub fuels? 

\parttitle{Results} %if any
We found that canopy traits are more important in determining flammability than leaf traits; total dry mass per 70\,cm branch and canopy density together were the best predictors of temperature integration in shoot flammability in shrub fuels. Furthermore, we found that shoot flammability in shrub fuel is mostly described by a single axis, represented by flammability metrics related to duration.

\parttitle{Conclusions}
This finding illustrates the potential for incorporating canopy traits in fire behavior models and might improve the understanding of fire-fuel feedback in the shrublands of Texas.

\end{abstract}

%%%%%%%%%%%%%%%%%%%%%%%%%%%%%%%%%%%%%%%%%%%%%%
%%                                          %%
%% The keywords begin here                  %%
%%                                          %%
%% Put each keyword in separate \kwd{}.     %%
%%                                          %%
%%%%%%%%%%%%%%%%%%%%%%%%%%%%%%%%%%%%%%%%%%%%%%

\begin{keyword}
\kwd{Shoot flammability}
\kwd{Canopy traits}
\kwd{Leaf traits}
\kwd{Temperature integration}
\kwd{Ignitibility}
\end{keyword}

% MSC classifications codes, if any
%\begin{keyword}[class=AMS]
%\kwd[Primary ]{}
%\kwd{}
%\kwd[; secondary ]{}
%\end{keyword}

\end{abstractbox}
%
%\end{fmbox}% uncomment this for twcolumn layout

\end{frontmatter}
%\documentclass{ttuthes2007}



%%%%%%%%%%%%%%%%%%%%%%%%%%%%%%%%%%%%%%%%%%%%%%%%%%
%Spacing -- Do you want double or one-and-a-half?%
%%%%%%%%%%%%%%%%%%%%%%%%%%%%%%%%%%%%%%%%%%%%%%%%%%
\doublespacing
%\onehalfspacing

\linenumbers

\section*{Background}
Fire is a dominant driver of change in many terrestrial ecosystems and flammability is a biological trait \citep{pausasandmoi2012flammability}. Vegetation fuels fire, which in turn affects vegetation \citep{bova2005linking, jones2006prediction, kavanagh2010way,o2010acute, michaletz2012moving, west2016experimental, lodge2018xylem, bar2019fire}
and understanding plant flammability is essential to understanding vegetation-fire feedbacks in fire-prone ecosystems \citep{pausas2012fire, pausas2017flammability}. Fuels, fires, and fire effects are heterogeneous over space \citep{gagnon2010does, o2018advances}, yet many studies investigating the effect of fire on vegetation take place after fires, with little information on fire behavior \citep{o2018advances}. \citet{schwilk2011scaling} demonstrated that it is possible to establish a link between characteristics of individual species  and fire severity in landscape level. Therefore, trait-based flammability studies are an important contribution to understanding how fire shapes ecosystems and species persistence in fire-prone ecosystems \citep{pausas2012fire, pausas2017flammability}. Different plant species and communities burn differently, and a previous study suggested that vegetation-fire feedback in crown-fire ecosystems is ecosystem specific \citep{pausas2004plant}. Fuel varies structurally and many plant traits can influence flammability. 
However, although we have been witnessing an increasing trend of high-severity wildfires in many parts of the world \citep{miller2012trends, dennison2014large, weber2020spatiotemporal, salguero2020wildfire}, a mechanistic understanding of the canopy traits-flammability relationship is still in an early stage.

Plant flammability is the general capacity of vegetation to burn. Different physical and chemical characteristics of plants affect how susceptible they are to burning \citep{bond1996fire}. It is well-recognized that plants' have both inter and intra-specific differences in their flammability \citep{pausas2012fire, battersby2017exploring, cui2020shoot, cui2022intraspecific} and also differ in growth form \citep{calitz2015investigating, jaureguiberry2011device, zanzarini2022flammability, potts2022growth, cui2020shoot, wyse2016quantitative}. However, measuring the flammability of plants on an individual scale and predicting fire behavior at larger scales is challenging. One complication is that the concept of flammability is subjective \citep{gill2005flammability} and methodological differences across studies make comparisons difficult. Historically, flammability was considered as consisting of four different components: ignitibility (the ability to start fire), combustibility (the mass loss rate of fuel per unit of time), consumability (the amount of fuel consumed or burned), and sustainability (the amount of time flame or combustion lasted) \citep{anderson1970forest, martin1993assessing}. However, empirical evidence suggests that these metrics are inter-correlated and most work indicates that flammability has three major dimensions: ignitability, total heat release, and fire spread rate \citep{pausas2017flammability}. These dimensions may be independent of one another for a particular scale but might be correlated at other scales. For example, past work in surface fire systems suggested that the flammability of leaf litter varies along two largely orthogonal dimensions, total heat release and flame spread rate \citep{scarff2006leaf, de2012leaf,cornwell2015flammability}. More recently, \citet{prior2018conceptualizing} suggested that these two main axes can be used to describe flammability at the community scale and successfully predict fire severity for a wide range of frequently measured flammability metrics.

The overall tendency of vegetation to catch and spread fire is strongly influenced by the flammability of dominant plant species. Therefore, species--specific flammability is a fundamental element in determining the ecological effects of fire \citep{bond1995kill, lavorel2002predicting, bond2005fire, wyse2018shoot}. However, studies of the traits that determine a plant's flammability, particularly at an individual scale, and, consequently, its unique contribution to the fire regime in a landscape, are uncommon \citep{jaureguiberry2011device, schwilk2015dimensions, pausas2017flammability}. Part of the reason is the lack of comparable, cost-effective protocols to measure the flammability of whole plants over large numbers of species \citep{jaureguiberry2011device}. Burning the whole plant would be more insightful to understanding the plant's flammability, but it is expensive and logistically challenging \citep{jaureguiberry2011device, pausasandmoi2012flammability}. Therefore, many flammability studies included burning individual leaves to predict canopy flammability and fewer considered the geometry of plant parts which has a great influence on fire behavior \citep{schwilk2003flammability, madrigal2012evaluation, pausas2012firesulex, calitz2015investigating, gao2018grass}. Moreover, recent studies suggest that measuring flammability by burning entire shoots is better at predicting plant flammability than burning leaves \citep{alam2020shoot} and individual leaf flammability is a poor predictor of crown fire behavior \citep{fernandes2012plant}. As a consequence, it is essential to use methods that measure the flammability of whole or partial plant canopies which retains the special arrangement of plant parts and make measurements reproducible, and can be more efficient in predicting 
patterns of wildland fire.



The plant traits that influence plant flammability are different on different scales \citep{pausas2017flammability}. For example, at the plant organ level such as within  the leaf, leaf morphology and chemistry determines the way leaf burns \citep{anderson1970forest, owens1998seasonal, schwilk2011scaling, pausas2016secondary, guerrero2021leaf, ganteaume2021volatile,alam2020shoot}. While some traits, like the presence of volatile compounds and high leaf dry matter content, increase leaf flammability, others, including high moisture content, and thick leaves, can decrease flammability. At the individual scale, plant architecture, such as fuel compactness and branching patterns is %are one of 
the most important factor in plant flammability \citep{schwilk2003flammability, madrigal2012evaluation}. For example, in denser canopies, heat transfer enables the fire to readily move from one branch to another \citep{bond1996fire}. Moreover, live fuel initially acts as a heat sink, and the amount of heat needed to substantially dry the live fuel as it begins to burn varies between %for the
leaf and stem. Therefore, the canopy's leaf cover is  also a key factor in crown fire \citep{ray2005micrometeorological}. However, the importance of canopy traits in shoot level flammability has yet to be tested \citep{alam2020shoot}.


Architectural branching designs vary among plants \citep{halle2012tropical} and plant architecture can be shaped by different selective pressures \citep{danell1994browseeffects, schwilk2003flammability} and interacts with other morphological and life history traits \citep{ackerly1998leaf, schwilk2001flammability,archibald2003growing}. A recent study suggests that considering the dissimilarities in branching patterns can improve plant scaling models \citep{bentley2013empirical}. Moreover, plant shoots as phytomers are influenced by the interaction between genotype and environmental pressure \citep{mcsteen2005shoot, wang2008molecular}. Therefore, it is reasonable to treat a terminal branch from a given plant sample as a canopy trait rather than representative of the total fuel load of that plant.

Our goal was to investigate the importance of traits in controlling the shoot flammability of native shrubs of Texas. 
We burned a total of 116 samples, 3-12 individuals per species across 16 common shrub species of Texas to measure flammability. We measured four common canopy traits: total dry mass per 70\,cm, canopy density, canopy moisture content and leaf: stem, as well as four common leaf traits: LMA, leaf area per leaflet, leaf moisture content, and leaf length per leaflet. We used these data to answer two questions: 1). Are canopy traits more important than leaf traits in canopy flammability? 2). Are heat release and flame spread independent axes of flammability in shrub fuels? Our research, investigating the importance of structural traits of native shrubs on shoot--level flammability might help improve the understanding of the shrub fuel-fire feedback in the shrublands of Texas.

\section*{Methods}
\subsection*{Site selection}
The species were selected from the shrub species found on the properties we had access to collect samples from during our preliminary visit to those properties. Those private properties were the part of Texas Ecological Laboratory  \url{https://texasecolabprogram.org/} program which provides tax benefits to landowners who allow research on their property.  To collect samples, we traveled to fourteen different properties in eleven counties in Texas (Bandera, Edwards, Menard, Duval, Uvalde, Kendall, Bastrop, Dickens, Van Zandt, Real, Bell).   
Plant samples were collected during summer (late May -- early August) 2022 from 16 native shrub species. The number of studied species from each property was maximized in light of the potential variance in flammability within species. However, only \emph{Diospyros texana}, one of Texas' most prevalent shrubs, was found at all eight properties. There were five properties where the species \emph{Juniperus ashei, Mahonia trifoliolata, Sophora secundiflora}, and \emph{Rhus virens} were found. Four properties contained three species (\emph{Senegalia wrightii, Prosopis glandulosa, and Ilex vomitoria}), while three properties contained \emph{Sarcomphalus obtusifolia} \citep{hauenschild2016phylogenetic} and \emph{Calicarpa americana}. The species \emph{Juniperus virginiana} and \emph{Forestiera pubescens} were found in two properties, but the remaining species were only found in one property each. Due to the challenge of differentiating \emph{Senegalia wrightii} from \emph{Senegalia greggii} without flowers during sampling, we treated them as a single taxon. We recorded the coordinates of the collection site by GPS, took pictures of each sample  and collected specimens of unidentified shrubs.
The sampled individuals were chosen from those who were in good health and did not appear to be under water stress. We collected three samples from each mature, healthy individual. Two of them comprised a pair of 70\,cm long shoot samples. These were chosen to be visually similar, healthy-looking terminal branches. We chose paired samples to allow one to be used later in flammability trials and the other to be used for destructive leaf and stem measurements. The third sample provided leaves for leaf trait measurements. The destructive measurements shoot sample was cut into sections, sealed inside a plastic food storage bag, and placed on ice in a cooler for later measurements. The shoot sample destined for burning trials was kept intact and placed inside a large sealable plastic bag (Ziploc BIG BAGS, SC Johnson, 60 \,cm × 82 \,cm × 18 \,cm). The leaves sample was placed in a small sealable plastic bag.

A paper towel, saturated with water was placed inside the plastic bags to maintain 100 \% relative humidity inside the bag and eliminate further drying after cutting. To avoid puncturing the plastic bag, we used a large cotton towel to wrap samples from species with spines, needles, or thorns. All the samples were placed inside an insulated cooler on ice and transported to Texas Tech University within five days of collection.  
At the lab, the intact shoot samples were left to bench dry for 36 hours in the lab before burning trials  as per established methods \citep{wyse2016quantitative}. A data logger (HOBO MX2300 temp/RH, Onset, Bourne, MA) recorded the temperature and humidity in the lab during the drying period at one second intervals.

\subsection*{Trait measurements}
The sampled branch assigned to destructive harvests was used for measuring the leaf and stem biomass. This branch was separated into leaves and stems for separate drying and weighing. The leaves, including petiole and rachis, were separated from the stem and their fresh mass was measured before placing them in a paper bag for drying. Following the measurement of the fresh mass, the leaves and stems were each dried at $65^{\circ}$C for at least 48 hours before weighing. Canopy density was measured during the drying period of the burned sample. We visually assigned each branch as approximating one of three possible geometric shapes: cylinder, truncated cone, or the combination of two truncated cones connected at their wide end. For those samples assigned the cylinder shape, we measured the diameter of the branch at three different positions, averaged those measurements, and calculated the volume according to the formula of the volume of a cylinder, see Equation \eqref{eq:volume_of_cylinder}. For those samples assigned the truncated cone shape, we measured the diameter at both ends and calculated it according to the appropriate equations, see Equation \eqref{eq:truncated_cone}. For those samples assigned the double truncated cone shape, we measured the maximum diameter, the distance of the maximum diameter from the distal and proximal ends of the branch, and the diameters at the distal and proximal ends. We then calculated the combined volume of the truncated cones,
see Equation \eqref{eq:double_truncated_cone}. Canopy density was calculated by dividing the total dry mass of the sample by the canopy volume.

\begin{equation}
\label{eq:volume_of_cylinder}
V = \pi r^2h
\end{equation}


\begin{equation}
\label{eq:truncated_cone}
V = \frac{1}{3}\pi h(r_1^2 + r_1r_2 + r_2^2)
\end{equation}

\begin{equation}
\label{eq:double_truncated_cone}
V = \frac{1}{3}h_1 \pi( r_{1}^2 +  r_{2}^2 +  r_{1} r_{2}) + \frac{1}{3}(h_1 - 70) \pi (r_{2}^2 +  r_{3}^2 +  r_{2} r_{3})
\end{equation}
The leaflet of compound leaves is functionally similar to a simple leaf
\citep{perez2016corrigendum}. Therefore, we used the leaflet for compound leaves to measure the leaf traits. Leaf area per leaflet, leaf length per leaflet, and leaf mass per area (LMA) were measured on the leaf samples. We used five leaflets for the majority of the species. For species with small leaflet the number of leaflets was not fixed. We used a ruler scale to measure each leaflet's length, including the petiole, and then we weighed the leaves both when they were fresh and then after 48 hours of 65 $^{\circ}$C oven drying. A Licor 3100 leaf area meter, which can measure the surface area of leaves, was used to measure the total fresh leaflet area. Each time before measurements, the leaf area meter was calibrated. The leaflet was positioned so that it lies flat on the bed and does not overlap any other leaflets.   
LMA was then calculated on  dry mass basis. The projected and total leaf area is different for non-flat leaves \citep{perez2016corrigendum, cornelissen2003handbook}. Therefore, we measured the leaf area of the three \emph{Juniperus} species manually. We assumed that the scale-like leaves are cylindrical in shape. The length and width of the leave were treated as the height and diameter of a cylinder respectively. We used ten leaves for each sample and I measured the diameter and length of each leaf with slide calipers and a meter scale respectively. Based on the formula for calculating a cylinder's surface area, see Equation \eqref{eq:cylindder_surface}, the area of the leaves was calculated. The 10 leaves were weighed fresh and then dried in an oven for 48 hours at 65 $^{\circ}$C to obtain the dry weight.
\begin{equation}
\label{eq:cylindder_surface}
S = 2\pi r(r+h)
\end{equation}

\section*{Flammability trials}

We conducted flammability trials using a specially-built device that allows standardized measurements of the canopy flammability of branch portions up to 70\,cm long \citep{jaureguiberry2011device}. The chosen procedure calls for preheating the 70\,cm branch on the grill (84 x 55 \,cm) for two minutes before lighting the sample with a blowtorch. The grill was heated by a flame created by propane gas. Preheating is crucial for simulating wildfires in nature, because live fuel first acts as a heat sink after being exposed to tremendous heat before drying out and eventually reaching a flash point when they begin to burn. In contrast to the adopted procedures, where they used a temperature gauge connected to the grill thermometer to monitor the grill temperature \citep{jaureguiberry2011device}, we used two black aluminum discs  suspended from the wind protection shield and spaced approximately 19 \,cm (the center of the discs) apart from the grill's surface. This method is similar to that employed by \citet{gao2022burn} to measure heat transfer. Before placing the samples on the grill, the temperature of the discs was measured and the heat output of the propane burner was adjusted to keep these disks within a ranged from 130.25 to 163.30 $^{\circ}$C, with a mean of 148.79 $^{\circ}$C temperature  during the experiment. The two minute preheating period before ignition is intended to equalize air temperature during the burn trial, however wind speed could influence burning and we used
a Kestrel 3000 pocket weather meter (Nielsen–Kellerman, Co.,Chester, PA, USA) to
measure the maximum wind speed before each trial. There are three missing values for wind speed for three samples, and we averaged the values based on trials for those samples. 
The pre and post--burning temperatures of each trial were also recorded to quantify the heat release (J). We also measured the flame height using a meter scale that was mounted to the flammability apparatus.


A few leaves (range: 0.044 -- 2.380 g, mean: 0.408 g) and a twig with leaves (range: 0.0796 -- 8.177 g, mean: 1.11 g) were detached from the sample to be burned right before burning in order to calculate the leaf moisture content and canopy moisture content respectively. The leaves and twig with leaves were weighed fresh and then dried in an oven for 48 hours at 65 $^{\circ}$C to obtain the dry weight and calculated the leaf moisture content and canopy moisture content on a dry mass basis. We measured the fresh weight of each sample using a portable balance (0.1 g readability) immediately before burning. We then laid the sample horizontally on the grill for two minutes for preheating. A Fluke 572-2 IR remote infrared thermometer, which can measure temperatures up to 900 $^{\circ}$C, was used to measure the aluminum discs' precombustion temperatures during the pre-heating phase. Some preliminary flammability trials showed that some samples from the most flammable species didn't ignite within the allotted 10 second ignition period, therefore, we kept the blow torch on until a sample caught fire and treated the ignitibility as ignition delay \citep{anderson1970forest}.  
 

We used a thermocouple recording system to keep track of the flame temperatures during the burning trials. The flame temperatures were continually collected at three sites at intervals of one second using the most widely used, K-type thermocouple sensors \citep{mcgranahan2020inconvenient} attached to data loggers. An assistant continuously monitored the highest flame height for each trial and measured it using the meter scale. When the flame was extinguished, we recorded the post-burn temperature of the discs, the time it took to ignite, and the overall burning time in seconds with a stopwatch. After the burn, we calculated the percentage consumed and weighed the leftover biomass. To reduce observer error, the proportion of the volume burned was visually estimated and agreed upon by at least two people. Those samples that caught fire during preheating were counted as zero seconds as ignition delay. Each trial's average gas flow from the propane gas cylinder was 20.35\,$g min^{-1}$. Three flammability metrics: temperature integration over 100$^{\circ}$C ($^{\circ}$C.s), duration over 100$^{\circ}$C, and peak temperature were measured using thermocouples data.

\section*{Data analysis}

All data analysis was conducted in R, version 4.2.2 \citep{R}. 
Flammability includes separate measurements that do not necessarily correlate with one another, therefore the first step before further analysis was to use principle components analyses (PCA) to transform the measured variables into orthogonal axes of variation. The measured variables were in different units. Therefore, the correlation matrix in the \pkg{prcomp} function was specified. In terms of fire ecology, it is not just the amount of heat generated during a fire that is important, but also the duration \citep{mcgranahan2020inconvenient}, because the thresholds for temperature and time-related mortality driven by fire vary for different organisms \citep{nelson1952observations,vines1968heat, bond1983dead, hengst1994bark,pinard1997fire,lawes2011bark, pingree2019myth} due to the avoidance–attraction traits for fire \citep{schwilk2001flammability, archibald2019unified}. Moreover, seasonal variations also affect how long a plant must be exposed to a certain temperature before it is killed by fire \citep{wright1970method}. As a result, the response variable temperature integration over 100 ($^{\circ}$C.s), which represented the sum of the average temperatures over each second during the same period, often used as a proxy measurement of heat release, was chosen \citep{gao2018grass, mcgranahan2020inconvenient}. Moreover, the effect of canopy traits on ignitibility has not been well documented for shoot--level flammability. Therefore, the ignition delay was also selected as a response variable. Due to the hierarchical structure of the data, to determine the importance of traits on flammability, we built linear mixed effects models with genus as a random variable since flammability is phylogenetically conserved \citep{cui2020shoot} and we assumed that morphologically similar species would behave similarly in a wildfire. However, all the species in this study except the three species of \emph{Juniperus} taxon,
are morphologically distinct, and they were treated as individual taxon. To avoid the  Type I error and an overestimation of the amount of variance explained that could arise from forward selection method \citep{blanchet2008forward}, initially, two separate global linear mixed effects models were built, where each type of trait with two-way interaction were the fixed effects and the temperature integration was the response variable. The \pkg{lmer} function from the \pkg{afex} package \citep{singmann2015packageafex,afexluke2017evaluating} was used to perform the mixed effects model. 
To check the collinearity among predictor variables and make the fixed effects comparable, we created the correlation matrix and standardized the independent variables as z-score. However, the log transformation of the response variable to satisfy the normal error assumption can result in different coefficient estimations, frequently unrelated to the original data, can alter the relationship between the original variables, and biased error variance estimations of the coefficient particularly for response variable having zero \citep{mccuen1990problems,packard2008model,o2010not, changyong2014log, st2018count}. And therefore, there was no transformation applied to the response variable. The Kendall rank correlation coefficient between leaf area per leaflet and leaf length per leaflet is 0.65 (see \textbf{supplementary information}), and we decided to use only the leaf length per leaflet in the model because previous flammability studies used leaf length per leaflet to predict flammability \citep{alam2020shoot}. 
The best model from all the subsets of the global model for each type of trait was selected by the Akaike information criterion ($AIC_{c}$). The automated model selection was performed by the \pkg{MuMin} package \citep{barton2015packagemumin}. We compared the best leaf traits model against the best canopy traits model according to log-likelihoods. After selecting the best predictors for each type of trait, the mixed effects models with best leaf and canopy traits were built with the \pkg{lme4} package in R \citep{bates2009package} using maximum likelihood estimation in order to evaluate the significance of fixed effects. We used the Anova() function from the \pkg{car} package \citep{fox2013hypothesis} to test the significance of predictors. Type  2  sums  of  squares  were  calculated  for testing the hypothesis \citep{langsrud2003anova}. To avoid unacceptable type 1 error, estimated degrees of freedom of residuals and p-values were calculated using the Kenward--Roger approximation \citep{kenward1997small}. All the plots were generated by \pkg{ggplot2} and \pkg{factoextra} package \citep{wickham2016packageggplot2, kassambara2017packagefactoextra}.



\section*{Results}

\subsection*{Flammability axes}

According to the principle component analysis, the first two principal components accounted for 78.6\% of the variation of all the measured flammability metrics. The first axis captured 69.1\% of the total variation (Figure \ref{fig:pca-plot}). The temperature integration in the first axis received the most contributions and six flammability traits had a strong positive correlation with one another (loadings: temperature integration over 100 $^{\circ}$C = 0.38, flame duration = 0.37, duration over 100C $^{\circ}$C = 0.37, mass-consumed = 0.37, peak temperature  = 0.36, percentage of volume burned = 0.35). The second axis of PCA explained 9.5\% of the total variation and the ignition delay and heat release (j) both contributed the most in the second principle component and were negatively correlated with each other (loading: ignition delay = -0.71, heat release (j) = 0.59). 


\begin{figure} 
    \centering
    \includegraphics[width =\textwidth]{../../results/pca_plot.pdf}
    \caption[Principle components results]{\label{fig:pca-plot} Principle component analysis biplot of nine flammability traits with their abbreviations (ID = Ignition delay, FD = Flame duration, TI = Temperature integration over 100 $^{\circ}$C, Duration over 100 $^{\circ}$C, MC = Mass consumed, VB = Volume burned, PT = Peak temperature, FH = Flame height, HR = Heat release (j)) with their quality of representation as $cos^2$ (squared coordinates). A high $cos^2$ indicates a good representation of the variable in the principle component and a low $cos^2$ of a variable indicates less importance in the principle component.}
  \end{figure}

\subsection*{Effect of canopy traits and leaf traits on temperature integration}
The automated model selection result shows that, for canopy traits, the model with total dry mass per 70\,cm and canopy density without interaction (Table \ref{tab:canopy_models}) was the best fit to predict temperature integration and \MakeUppercase{lma} alone was the best fit as predictor among all the leaf traits (Table \ref{tab:leaf_models}). The model with canopy traits was better at fitting the data set compared to leaf traits ($AIC_{c}$ value: best canopy traits model = 2491.8, best leaf traits model = 2585.0). We built another linear mixed effect model using the best canopy traits, mean pre--burning disc temperature and wind speed as  covariate to see whether adding pre--burning temperature and wind speed during trials improved the model or not. We found that, adding pre--burning temperature and wind speed didn't improve the model ($AIC_{c}$ value: best canopy traits model = 2491.8, pre--burning temperature and wind speed model = 2494.7).
%\begin{landscape}
\begin{table}
  \centering
  \caption{Model selection results for top four models from the global linear
    mixed effect model for canopy traits for temperature integration with
    corresponding $AIC_{c}$ value and model weight. The best model is highlighted in bold.}
  %\vspace{0.5 cm}
  \begin{tabular}{lrrr}
    \toprule
    \textbf{Fixed terms} & \textbf{$AIC_{c}$} & \textbf{Weight}\\
    \midrule
    \textbf{total dry mass + canopy density}    & \textbf{2491.8} &  \textbf{0.204}\\
    total dry mass + canopy density + leaf stem mass ratio \\+ canopy density: leaf stem mass ratio & 2492.0  & 0.184 \\
    total dry mass + canopy moisture content + canopy density   & 2492.0   & 0.139 \\ 
    total dry mass + leaf stem mass ratio + canopy density  & 2493.0 & 0.112  \\
    \bottomrule
  \end{tabular}
  \label{tab:canopy_models}
\end{table}
%\end{landscape}



\begin{table}
  \centering
  \caption{Model selection results for top four models from the global linear
    mixed effect model for leaf traits for temperature integration with
    corresponding $AIC_{c}$ value and model weight. The best model is highlighted in bold.}
  %\vspace{0.5 cm}
  \begin{tabular}{lrr}
    \toprule
    \textbf{Fixed terms} & $AIC_{c}$ & \textbf{Weight}\\
    \midrule
    \textbf{LMA} & \textbf{2585.0} & \textbf{0.414} \\
    LMA + leaf moisture content & 2587.1     & 0.143    \\
    LMA + leaf length per leaflet                                 & 2587.2 & 0.139    \\
    LMA + leaf length per leaflet + leaf length per leaflet:LMA                              & 2587.8     & 0.101    \\
    \bottomrule
  \end{tabular}
  \label{tab:leaf_models}
\end{table}

After getting the best predictors for temperature integration for each type of traits from the initial model selection, we tested the significance of the relationship  between predictors and the response variable. Both the total dry mass per 70\,cm and canopy density had a significant positive linear effect  
on temperature integration above 100 $^{\circ}$C. Marginal and conditional $R^2$ for the model with total dry mass per 70\,cm and canopy density without interaction was  0.63 and 0.75 respectively (total dry mass per 70\,cm: p $<$ 0.001, canopy density: $p = 0.011$) (see \textbf{supplementary information}) (Table \ref{tab:fandpstatfortemp}). LMA had also a significant positive effect on temperature integration above 100 $^{\circ}$C. Marginal  and conditional $R^2$ for the model with LMA was 0.05 and 0.50 respectively (LMA: $p = 0.020$), see \textbf{supplementary information}. The majority of the canopy traits among all the measured canopy traits had higher values for the species from the \emph{Juniperus} taxon, especially \emph{Juniperus ashei} and \emph{Juniperus pinchotii} and the measurements of the leaf traits was also error prone due to having the non--flat leaves for those species. In order to understand how this taxon influenced the results, we removed the species from the \emph{Juniperus} taxon from the analysis and then examined the impact of those traits on the remaining species. After removing the \emph{Juniperus} species from the analysis, the total dry mass per 70\,cm still had a strong positive effect ($p $<$ 0.001$)(Figure \ref{fig:dm-tempint}), but neither canopy density nor LMA have any significant effect on temperature integration for the remaining species (canopy density: $p = 0.367$, LMA: $p = 0.901$), see \textbf{supplimentary information}. 

\begin{table}
\centering
\caption{Mixed effect model coefficients and ANOVA table for the best predictors for each type of traits for temperature integration over 100 $^{\circ}$C. The linear mixed model was fitted with \pkg{lmer} function from \pkg{lme4} package \citep{bates2009package}. The \pkg{car} package in R \citep{fox2013hypothesis} was used to calculate the estimated degrees of freedom of residuals, F and p-values using the Kenward-Roger approximation \citep{kenward1997small}. All the independent variables were standardized  as z-score and the response variable was log transformed.}
\vspace{0.5 cm}
\begin{tabular}{lrrrr}
  \hline
 &  Estimate & F  & Df.res & Pr($>$F) \\ 
  \hline 
  total dry mass (g) & 14075.4 & 103.0  & 112.9 & \textbf{$<$0.001} \\ 
  canopy density (g/{$cm^3$}) & 3054.8 & 6.7  & 106.6 & \textbf{0.011} \\ 
  LMA & 4810.5 & 5.5 &  101.4 & \textbf{0.020} \\ 
   \hline
\end{tabular}
\label{tab:fandpstatfortemp}
\end{table}

\begin{figure}[ht]
    \centering
    \includegraphics[width = \textwidth]{../../results/total_dry_mass.pdf}
    \caption[Dry mass effect on temperature integration]{\label{fig:dm-tempint}Relationship between total dry mass and temperature integration above 100 $^{\circ}$C without three species from \emph{Juniperus} taxon. The line indicates the best-fitted linear mixed model with genus as a random intercept effect ($p $<$ 0.001$). The independent variable was standardized as z-score. Small points in the background are individual observations, and large points are the taxon's mean.}
\end{figure}




\subsection*{Effect of canopy and leaf traits on ignition delay}
For ignition delay, the global linear mixed effects model were built without any interaction term  to avoid over fitting problem since ignition source \citep{anderson1970forest, madrigal2012evaluation} and many unmeasured traits like volatile compounds can influence ignitibility. For canopy traits, the model with canopy density and canopy moisture content was the best fit to predict ignition delay (Table \ref{tab:canopy_models_ignition_delay}) and LMA and leaf moisture content was the best fit as predictor among all the leaf traits (Table \ref{tab:leaf_models_ignition_delay}). The model with canopy traits was better at fitting the data set compared to leaf traits ($AIC_{c}$ value: best canopy traits model = 579.7, best leaf traits model = 585.4). We built another linear mixed effects model using the best canopy traits,  mean pre--burning disc temperature and wind speed as  covariate to see whether adding pre--burning temperature and wind speed improved the model or not. We found that, adding pre--burning temperature and wind speed didn't improve the model ($AIC_{c}$ value: best canopy traits model = 579.7, pre--burning temperature wind speed model = 582.2).

\begin{table}
  \centering
  \caption{Model selection results for top four models from the global linear
    mixed effect model for canopy traits for ignition delay with corresponding
    $AIC_{c}$ value and model weight. The best model is highlighted in bold.}
  % \vspace{0.5 cm}
  \begin{tabular}{lrrr}
    \toprule
    \textbf{Fixed terms} & $AIC_{c}$ & \textbf{Weight}\\
    \midrule
    %% DWS; still does not fit. Need to fix or make a landscape table
    \textbf{canopy den. + moisture content}    & \textbf{579.7} & \textbf{0.425} \\
    canopy den. + moisture content + total dry mass   & 580.6   & 0.275 \\
    canopy den. + moisture content + leaf stem mass ratio       & 581.6   & 0.166  \\
    canopy den. + moisture content + leaf stem mass ratio + total dry mass & 582.7  & 0.095 \\
    \bottomrule
  \end{tabular}
  \label{tab:canopy_models_ignition_delay}
\end{table}

\begin{table}
  \centering
  \caption{Model selection results for top four models from the global linear
    mixed effect model for leaf traits for ignition delay with corresponding
    $AIC_{c}$ value and model weight. The best model is highlighted in bold.}
  \begin{tabular}{lrrr}
    \toprule
    \textbf{Fixed terms} & $AIC_{c}$ & \textbf{Weight}\\
    \midrule
    \textbf{LMA + leaf moisture content}    & \textbf{585.4} & \textbf{0.596} \\
    Leaf length per leaflet + LMA + leaf moisture content & 586.9          & 0.288          \\
    LMA                                                   & 589.8          & 0.067          \\
    Leaf length per leaflet + LMA                         & 591.9          & 0.023          \\
    \bottomrule
  \end{tabular}
  \label{tab:leaf_models_ignition_delay}
\end{table}
After testing the significance of the predictors on ignition delay, we found that all the predictors from the best canopy (Figures \ref{fig:canopy_density_ig_delay} and \ref{fig:canopy_moisture_ig_delay}) and best leaf traits (Figures \ref{fig:lma_ig_delay} and \ref{fig:leafmc_ig_delay}) model had a strong positive effect on ignition delay. The marginal and conditional $R^2$ for the model with canopy density and canopy moisture content was 0.20 and 0.31 respectively (canopy density: $p < 0.001$, canopy moisture content: $p = 0.005$), and the marginal and conditional $R^2$ for the model with LMA and leaf moisture content was 0.14 and 0.27 respectively (LMA: $p = 0.003$, leaf moisture content: $p = 0.014$). (Table \ref{tab:fandpforig_delay})

\begin{table}[ht]
  \centering
  \caption{Mixed effects model coefficients and ANOVA table for the best
    predictors among the measured canopy and leaf traits for ignition delay.
    The linear mixed model was fitted with \pkg{lmer} function from \pkg{lme4}
    package \citep{bates2009package}. The \pkg{car} package
    \citep{fox2013hypothesis} was used to calculate the estimated degrees of
    freedom of residuals, F and p-values using the Kenward-Roger approximation
    \citep{kenward1997small}. All the independent variables were standardized
    as z--scores.}
  % \vspace{0.2cm}
  \begin{tabular}{lrrrr}
    \toprule
    & Estimate & F & Df.res & Pr($>$F) \\ 
    \midrule
    canopy density (g/{$cm^3$}) & 1.1 & 14.5  & 108.6 & \textbf{$<$0.001} \\ 
    canopy moisture content (\%) & 0.9 & 8.4 & 63.2 & \textbf{0.005} \\ 
    LMA & 1.0 & 9.3  & 59.4 & \textbf{0.003} \\  
    leaf moisture content (\%) & 0.8 & 6.4  & 62.3 & \textbf{0.014} \\ 
    \bottomrule
  \end{tabular}
  \label{tab:fandpforig_delay}
\end{table}
After removing \emph{Juniperus} from the analysis, none of the traits from the best models had significant effect on ignition delay other than leaf moisture content (canopy density: $p = 0.423$ , LMA: $p = 0.330$, leaf moisture content: $p = 0.048$), see \textbf{supplementary information}. Leaves are one of the first plant parts to catch fire. Therefore, since leaf moisture content and canopy moisture content are highly correlated (the kendall rank correlation coefficient is .64, see \textbf{supplementary information}), to minimize the number of models, the canopy moisture content was removed from the model that was used to test the effect of those traits after removing the three species from the \emph{Juniperus} taxon.

\begin{figure}[ht]
  \centering \includegraphics[width =
  \textwidth]{../../results/leaf_moisture_ignition.pdf}
  \caption[Leaf moisture content effect on ignition delay]{\label{fig:leafmc_ig_delay}Relationship between leaf moisture content and ignition delay without the three species from \emph{Juniperus} taxon. The line indicates the best--fitted linear mixed effects model with genus as random intercept effect ($p = 0.048$). The independent variable was standardized  as z-score. Small points in the background are individual observations, and large points are the taxon's means.} 
\end{figure}



\section*{Discussion}
We found that canopy traits are more important than leaf traits in shoot flammability in shrub fuels. Specifically, species with higher amounts of fuel per 70\,cm length and higher canopy density burned for longer period of time and produced more heat during burning. These observations are consistent with the importance of the spatial arrangement of plant parts such as the bulk density \citep{pausas2012firesulex} and twigginess \citep{potts2022growth} in small--scale flammability experiments. The higher amount of biomass per 70\,cm shoots burned for longer time and, in denser shoots the fire can easily move from one twig to another and eventually burned for longer time. These observations are consistent with the importance of canopy architecture in fire behavior at the individual scale \citep{schwilk2003flammability, madrigal2012evaluation} which demonstrate the potential of scaling up the fire behavior by burning shoots. 

Leaf mass per area (LMA), one of the most important traits for describing leaf economics \citep{wright2004worldwide} appeared to be the most important determinant of temperature integration among all the measured leaf traits in this study. LMA is related to leaf thickness \citep{niinemets1999research}, and species with high LMA generally have more dense leaf tissue \citep{poorter2009causes}. Like leaf dry matter content (LDMC) in shoot flammability \citep{alam2020shoot, potts2022growth}, this observation was also likely due to the denser leaf tissue, which allows for longer burning with greater intensity. This result is consistent with previous leaf flammability experiments \citep{krix2018landscape}. However, an additional step in the model selection results showed that after adding the LMA to the best canopy traits model didn't improve the model. Moreover, this relationship is not significance after removing the most flammable taxon from the analysis which further confirms that the temperature integration was mostly driven by canopy traits in this experiment.

We found that the two most important determinants of ignition delay for canopy traits were canopy density and canopy moisture content. The more compact the litter bed, the lesser the oxygen  \citep{scarff2006leaf, van2012species, engber2012patterns, de2012leaf, cornwell2015flammability}.  
Therefore, unlike litter flammability where the overall flammability is oxygen--limited \citep{schwilk2015dimensions}, in shoot--level flammability, to some extent, only the ignition of fuel might be oxygen limited. For the leaf traits, the thicker leaves took longer to ignite compared to the thinner leaves which is consistent with previous shoot flammability study \citep{alam2020shoot}.   
Unlike \citet{ mason2016fire} where they found that the rate of heat release is low for the thicker leaves during burning, in shoot flammability, some species may take longer to ignite but might produce more heat during burning. Moreover, \citet{anderson1970forest} and  \citet{madrigal2012evaluation} suggested that the ignition source can influence ignitibility. Therefore, for cases like this, for a constant heat for the source of ignition, particularly if the heat source produces less heat than the activation eneragy of the exothermic reaction, quantifying the ignitibility as ignition delay as suggested by \citep{anderson1970forest} might be an appropriate way to capture this phenomenon in shoot--level flammability experiment.

Aside from moisture content, none of the traits from the best model (the model with canopy density, LMA and leaf moisture content) significantly affected ignition delay after the most flammable taxon \emph{Juniperus} was excluded from the analysis. This finding implies that, ignition delay was mostly driven by moisture content in this experiment. Previous studies suggested that the ignition delay in shrub fuel is less than 15 seconds when the live fuel moisture content is less than 100\% but some species from \emph{Juniperus} taxon may take longer to ignite which is consistent with this study \citep{pellizzaro2007seasonal}. The maximum leaf moisture content of burning samples in this study was 108.05 \% in \emph{Forestiera pubescens} but the maximum ignition delay was 18 seconds for species from \emph{Juniperus} taxon, see \textbf{supplementary information}.
Moreover, some species from this study are well known for the volatile components in their leaves and some of them  ignited during the pre--heating period. Therefore, the ignitibility in shoot flammability might be sometimes difficult to determine by measuring only leaf functional and structural traits. 

Canopy density influences airflow in the canopy \citep{cionco1978analysis} and might be less effective for a single shoot to understand the impact of the limitation of the airflow and how a large amount of biomass influences the fire spread rate. 
Therefore, future studies are required to investigate to what extent the geometry of a single shoot can capture the structural variation in three-dimensional space in terms of fire behavior. Moreover, in some ecosystems where the forest floor is wet, the fire spread rate in the canopy is high-density limited  \citep{ray2005micrometeorological}. Therefore, although the leaf traits like leaf length per leaflet are less important in shoot-level flammability, in some ecosystems they might be influential in crown fire since species with larger leaves might be more aerated in the canopy, which affects the rate at which fires spread by permitting the flow of the hot, dry air. 

Texas and fire have a long history together \citep{moir1982firehistory, stambaugh2011firehistory,stambaugh2014historicalfirehistory,smeins2005historyoffire1}, and there were fire adapted species in this study (e.g. \emph{Prosopis glandulosa}) \citep{wright1976effect}. 
Moreover, all of the most flammable shrubs in this experiment are cone and seed--bearing plants 
and study suggests that fire influences seed coat chemistry \citep{mcinnes2022doesseedcoatchemistry} and in general, most flammable plants are reseeders \citep{midgley2011pushingreseeders}. Therefore,  the observed strong positive effect of total dry mass per 70\,cm length and canopy density on flammability might be selected to spread fire to the surrounding plants because if fires kill nearby, less flammable neighbors and also increases fecundity, flammability may improve inclusive fitness \citep{bond1995kill}. However, those structural traits might also have other competitive advantages such as light harvesting. Moreover, Texas has a long history with grazing as well and herbivores can also influence the architecture of plants \citep{danell1994browseeffects}. A previous study regarding \emph{Juniperus ashei}, one of the most flammable species in this study, suggested that the enhanced flammability might be the response to herbivores \citep{owens1998seasonal} in these ecosystems. Moreover, a recent study also showed that \emph{Juniperus ashei} has an allelopathic relationship with \emph{Bouteloua curtipendula}, a native grass in this ecosystem \citep{young2009assessmentallelopathy}. Therefore, future studies are required to investigate whether it is a single evolutionary force or a combination of different selective pressure that drove the enhanced flammability in these ecosystem.

%%%%%%%%%%%%%%%%%%%%%%%%%%%%%%%%%%%%%%%%%%%%%%
%%                                          %%
%% Backmatter begins here                   %%
%%                                          %%
%%%%%%%%%%%%%%%%%%%%%%%%%%%%%%%%%%%%%%%%%%%%%%

\begin{backmatter}



\section*{Author's contributions}

To complete this work, all of the listed authors contributed equally.

\section*{Acknowledgements}
This research is funded by Texas Ecological Laboratory  \url{https://texasecolabprogram.org/} and Native Plant Society of Texas \url{https://npsot.org/wp/}. This research would not have been successful without the invaluable contributions of the following individuals during sample collection and burning: Sharandeep Chahal, Imtiaz Qavi, Muhammad Usman, Peter Eludini, Keely Ussery, Hampton Heath, Benjamin Harrington, Mark Bazemore, Ashlyn Sneed, Bittnah Kim, Adriana Botero, Caden M Casanova, Hector Maynes, Pawan Devkota,  Rakesh Singh, Yanni Chen,  Md Wahiduzzaman, Md Towhidul Islam Tareqe, Syed Ehsan Ar Rafi, Md Faiyaz Kabir, Mohammad Tarak Aziz, Shaikh Sadiqur Rahman, Golam Rabbani and Sowrav Barua.

\section*{Competing interests}
  The authors declare that they have no competing interests.

\bibliographystyle{ecology}
\bibliography{myref}
\end{backmatter}
\end{document}