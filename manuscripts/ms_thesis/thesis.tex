\documentclass{ttuthes2007}

%%%%%%%%%%%%%%%%%%%%%%%%%%%%%%%%%%%%%%%%%%%%%%
%Include any other add-on  packages you need:%
%%%%%%%%%%%%%%%%%%%%%%%%%%%%%%%%%%%%%%%%%%%%%%
\let\bibhang\relax % ttuthes2007 is not easily compatible with natbib

\usepackage{natbib}
\bibpunct{(}{)}{,}{a}{}{,}
\usepackage{amsmath,graphicx}
\usepackage[utf8]{inputenc}
\usepackage[T1]{fontenc}
\usepackage{lmodern}
\usepackage[labelfont=bf,labelsep=period]{caption}
%\usepackage{multirow}
%\usepackage{textcomp}
\usepackage[colorlinks=true,citecolor=black,linkcolor=black]{hyperref}
%\usepackage{enumitem}
\usepackage{lineno}
%\usepacakge{blindtext}
\usepackage{setspace}
\usepackage{booktabs}

%\usepackage[top=1in, left=1.5in, right=1in, bottom=1in]{geometry}

% Format for typesetting R packages
\newcommand{\pkg}[1]{\textsc{#1}} 

%%%%%%%%%%%%%%%%%%%%%%%%%%%%%%%%%%
%EDIT  (Running head--  REQUIRED)%
%%%%%%%%%%%%%%%%%%%%%%%%%%%%%%%%%%
\rhead{\small Texas Tech University, \textit{Azaj Mahmud}, May 2023}	%update your name and graduation date-year here

%%%%%%%%%%%%%%%%%%%%%%%%%%%%%%%%%%%%%%%%%%%%%%%
%Uncomment if the grad school doesn't like the%
%line under the  running head:                %
%%%%%%%%%%%%%%%%%%%%%%%%%%%%%%%%%%%%%%%%%%%%%%%
\renewcommand{\headrulewidth}{0pt}


%%%%%%%%%%%%%%%%%%%%%%%%%%%%%%%%%%%%%%%%%%%%%%%%%%
%Spacing -- Do you want double or one-and-a-half?%
%%%%%%%%%%%%%%%%%%%%%%%%%%%%%%%%%%%%%%%%%%%%%%%%%%
\doublespacing
%\onehalfspacing
%%%%%%%%%%%%%%%%%%%%%%%%%%%%%%%%%%%%%%%%%%%%%%%%%%%%%%%%%%%%%
%Leave the one you want uncommented.                        %
%In places where single-line-spacing is appropriate         %
%e.g, extended quotations, you can enclose the material     %
%in a singlespacing environment (with \begin{singlespacing} %
% ...  \end{singlespacing}                                  %
%%%%%%%%%%%%%%%%%%%%%%%%%%%%%%%%%%%%%%%%%%%%%%%%%%%%%%%%%%%%%


%%%%%%%%%%%%%%%%%%%%%%%%%%%%%%%%%%%%%%%%%%%%%%%%%%
%Other preamble stuff, e.g., theorem environments%
%or newcommands go here:                         %
% e.g.                                           %
%%%%%%%%%%%%%%%%%%%%%%%%%%%%%%%%%%%%%%%%%%%%%%%%%%
% \newtheorem{theorem}{Theorem}
% \newtheorem{proposition}[theorem]{proposition}
% \newtheorem{question}{Question}
% \newtheorem{conjecture}{Conjecture}


%%%%%%%%%%%%%%%%%%%%%%%%%%%%%%%%%%%%%%%%%%%%%%%%%%
%Start main document %
%%%%%%%%%%%%%%%%%%%%%%%%%%%%%%%%%%%%%%%%%%%%%%%%%%
\begin{document}

%%%%%%%%%%%%%%%%%%%%%%%%%%%%%%%%%%%%%%%%%%%%%%%%%%
%%%%%%%%%%%%%%%%%%%%%%%%%%%%%%%%%%%%%%%%%%%%%%%%%%
%%%%%%%%%%%%%%%%%%%%%%%%%%%%%%%%%%%%%%%%%%%%%%%%%%
%Title page -- Edit the spacing commands after   %
%each \\ if necessary                            %
%%%%%%%%%%%%%%%%%%%%%%%%%%%%%%%%%%%%%%%%%%%%%%%%%%
%%%%%%%%%%%%%%%%%%%%%%%%%%%%%%%%%%%%%%%%%%%%%%%%%%
%%%%%%%%%%%%%%%%%%%%%%%%%%%%%%%%%%%%%%%%%%%%%%%%%%
\begin{titlepage}
\vbox to  \textheight{
\begin{singlespacing}
  \begin{center}
  Shoot flammability of shrubs in Texas\\[15pt]
by\\[15pt]
Azaj Mahmud, B.S\\[15pt]
A Thesis\\[15pt]
In\\[15pt]
Department of Biological Sciences\\[15pt]
Intend to submit to the Graduate Faculty\\
of Texas Tech University in\\
Partial Fulfillment of\\
the Requirements for the Degree of\\[15pt]
Master's of Science\\[30pt]
Approved\\[15pt]
Dylan W Schwilk, Ph.D.\\
Chair of Committee\\[15pt]
Nathan Gill, Ph.D.\\[15pt]
Nicholas G. Smith, Ph.D.\\[15pt]
Mark Sheridan, Ph.D.\\
Dean of the Graduate School\\[30pt]
May 2023
\end{center}
\end{singlespacing}
\vfill}
\end{titlepage}

%%%%%%%%%%%%%%%%%%%%%%%%%%%%%%%%%%%%%%%%%%%%%%%%%%
%End of title page                               %
%%%%%%%%%%%%%%%%%%%%%%%%%%%%%%%%%%%%%%%%%%%%%%%%%%


%%%%%%%%%%%%%%%%%%%%%%%%%%%%%%%%%%%%%%%%%%%%%%%%%%
%%%%%%%%%%%%%%%%%%%%%%%%%%%%%%%%%%%%%%%%%%%%%%%%%%
%%%%%%%%%%%%%%%%%%%%%%%%%%%%%%%%%%%%%%%%%%%%%%%%%%
%Copyright page                                  %
%usage: \copyrightpage{year of appearance}{Name} %
%%%%%%%%%%%%%%%%%%%%%%%%%%%%%%%%%%%%%%%%%%%%%%%%%%
%%%%%%%%%%%%%%%%%%%%%%%%%%%%%%%%%%%%%%%%%%%%%%%%%%
%%%%%%%%%%%%%%%%%%%%%%%%%%%%%%%%%%%%%%%%%%%%%%%%%%
 % Copyright Page
%\frontmatter
\copyrightpage{2023}{Azaj Mahmud} 

%%%%%%%%%%%%%%%%%%%%%%%%%%%%%%%%%%%%%%%%%%%%%%%%%%
%end of copyright page                          %
%%%%%%%%%%%%%%%%%%%%%%%%%%%%%%%%%%%%%%%%%%%%%%%%%%

%%%%%%%%%%%%%%%%%%%%%%%%%%%%%%%%%%%%%%%%%%%%%%%%%%
%Start of frontmatter                            %
%You need this:                                  %
%%%%%%%%%%%%%%%%%%%%%%%%%%%%%%%%%%%%%%%%%%%%%%%%%%
\frontmatter

%%%%%%%%%%%%%%%%%%%%%%%%%%%%%%%%%%%%%%%%%%%%%%%%%%
%%%%%%%%%%%%%%%%%%%%%%%%%%%%%%%%%%%%%%%%%%%%%%%%%%
%%%%%%%%%%%%%%%%%%%%%%%%%%%%%%%%%%%%%%%%%%%%%%%%%%
%Acknowledgements                                %
%%%%%%%%%%%%%%%%%%%%%%%%%%%%%%%%%%%%%%%%%%%%%%%%%%
%%%%%%%%%%%%%%%%%%%%%%%%%%%%%%%%%%%%%%%%%%%%%%%%%%
%%%%%%%%%%%%%%%%%%%%%%%%%%%%%%%%%%%%%%%%%%%%%%%%%%
\chapter{\textbf{Acknowledgements}}
\noindent First and foremost, I want to sincerely thank Dr. Dylan Schwilk, my M.S. advisor, for his unwavering support. I couldn't have finished my thesis without his guidance and encouragement. For my M.S., I can't think of a better mentor and advisor.I also want to thank the other members of my thesis committee, Nathan Gill, Ph.D., and Nicholas G. Smith, Ph.D., for their support and time.

Despite the fact that we have shared our lab for a short period, I especially want to thank Dr. Xiulin Gao. I believe that without the R scripts in Dr. Gao's GitHub projects, I couldn't complete my analysis in R. A special thanks to Muhammad Usman , Keely Ussery , Hampton Heath, Benjamin Harrington , Mark Bazemore, Ashlyn Sneed, Bittnah Kim,  Adriana Botero and  Caden M Casanova. Without you guys, it would be impossible for me to collect the samples and do the burning.

I want to thank my labmate Peter Eludini for being a friend and a brother. I also like to thank all the memebers of Ecohealth lab. Spcecial thanks to Pawan Devkota and  Rakesh Singh for the field trip. Thank you Evan Perkowski, Ezinwanne Ezekannagha, Chatterjee, Snehanjan, Eve Gray, Monika Kelley, David Bowers, Isabella Beltran Triana, Rafael Freitas, Sara Bohi Goulart for your friendship and for making our lab home. Special thanks to Brad Posch, Kate Fuller,  Jeffrey Chieppa for your friendship. 

Special thanks to  Sharandeep Chahal. I couldn't have completed the sample collection in 2022 without your help. Special thanks to Imtiaz Qavi, Golam Rabbani, Sowrav Barua,  Md Wahiduzzaman,
Md Towhidul Islam
Tareqe, Syed Ehsan Ar Rafi, Md Faiyaz Kabir, Mohammad Tarak Aziz, Shaikh Sadiqur Rahman for your help and friendship.

I would like to thank the Bangladeshi community in Lubbock for their help. Last but not least, I'd  also like to thank my family and friends for helping me along the way. Special thanks to parents for believing in me. 
%%%%%%%%%%%%%%%%%%%%%%%%%%%%%%%%%%%%%%%%%%%%%%%%%%
%Table of Contents                               %
%%%%%%%%%%%%%%%%%%%%%%%%%%%%%%%%%%%%%%%%%%%%%%%%%%
\newpage
\begin{singlespace}
\tableofcontents
\end{singlespace}
%%%%%%%%%%%%%%%%%%%%%%%%%%%%%%%%%%%%%%%%%%%%%%%%%%
%Abstract                                        %
%%%%%%%%%%%%%%%%%%%%%%%%%%%%%%%%%%%%%%%%%%%%%%%%%%
\chapter{\textbf{Abstract}}
\noindent Flammability is a trait of land plants, and an understanding of this trait is important to comprehend the ecological and evolutionary consequences of fire on plant species in fire-prone ecosystems. Numerous plant traits can influence flammability. Although many studies have examined the flammability of individual leaves, fewer have examined how entire canopies behave as fuel because quantifying plant flammability by burning whole plants is expensive. With the development of a new plant flammability device that allows standardized measurement of the canopy flammability of portions up to 70\,cm long and in the light of a recent study regarding the effectiveness of burning shoots to predict plant flammability; I burned 70\,cm branches from at least three samples per species of 16 native shrubs of Texas and measured four canopy traits: total dry mass per 70\,cm, canopy density, leaf: stem (in dry mass basis) and canopy moisture content and four common leaf traits: leaf mass per area (LMA), leaf area per leaflet, leaf length per leaflet and leaf moisture content to answer two questions: 1) Are canopy traits more important than leaf traits in shoot flammability? 2) Are heat release and flame spread rate independent axes of flammability in shrub fuels? I found that canopy traits are more important in determining flammability than are leaf traits; total dry mass per 70\,cm branch and canopy density together are the best predictors of heat release in shoot flammability in shrub fuels. Furthermore, I found that shoot flammability in shrub fuel is mostly described by a single axis, represented by flammability metrics related to duration and temperature. This finding illustrates the potential for incorporating canopy traits in fire behavior models and might improve the understanding of fire-fuel feedback in the shrublands of Texas.


%%%%%%%%%%%%%%%%%%%%%%%%%%%%%%%%%%%%%%%%%%%%%%%%%%
%List of tables and list of figures              %
%%%%%%%%%%%%%%%%%%%%%%%%%%%%%%%%%%%%%%%%%%%%%%%%%%
\listoftables
\listoffigures

%%%%%%%%%%%%%%%%%%%%%%%%%%%%%%%%%%%%%%%%%%%%%%%%%%
%MAIN PART OF  DOCUMENT                          %
%%%%%%%%%%%%%%%%%%%%%%%%%%%%%%%%%%%%%%%%%%%%%%%%%%
\mainmatter

%%%%%%%%%%%%%%%%%%%%%%%%%%%%%%%%%%%%%%%%%%%%%%%%%%
%%%%%%%%%%%%%%%%%%%%%%%%%%%%%%%%%%%%%%%%%%%%%%%%%%
%%%%%%%%%%%%%%%%%%%%%%%%%%%%%%%%%%%%%%%%%%%%%%%%%%
%Start introduction chapter                      %
%%%%%%%%%%%%%%%%%%%%%%%%%%%%%%%%%%%%%%%%%%%%%%%%%%
%%%%%%%%%%%%%%%%%%%%%%%%%%%%%%%%%%%%%%%%%%%%%%%%%%
%%%%%%%%%%%%%%%%%%%%%%%%%%%%%%%%%%%%%%%%%%%%%%%%%%
\rhead{Texas Tech University, \textit{Azaj Mahmud}, May 2023}

% Should remove line numbers when submitting to grad school
\linenumbers
\renewcommand{\linenumberfont}{\normalfont\bfseries\large\color{black}}

\chapter{Canopy traits drive plant flammability: shoot flammability of Texas shrubs}

\section{Introduction}

Fire is a dominant driver of change in many terrestrial ecosystems and flammability is a biological trait \citep{pausas2012flammability}. Vegetation fuels fire, which in turn affects vegetation \citep{bova2005linking, jones2006prediction, kavanagh2010way,o2010acute, michaletz2012moving, west2016experimental, lodge2018xylem, bar2019fire}. Fire alters ecosystem processes \citep{debano1978effect, debano1998fire, grogan2000fire,wan2001fire, keane2008ecological, roces2022global,ojima1994long} and species compositions \citep{cochrane2003fire,cleary2004changes, laurance2003slow}. Different plant species and communities burn differently, and study suggests that vegetation-fire feedback in crown-fire ecosystems is ecosystem specific \citep{pausas2004plant}. 
Understanding plant flammability is essential to understanding vegetation-fire feedbacks in fire-prone ecosystems \citep{pausas2012fire, pausas2017flammability}. Fuels, fires, and fire effects are heterogeneous over space \citep{gagnon2010does}, yet many studies investigating the effect of fire on vegetation take place after fires, with little information on fire behavior \citep{o2018advances}. Therefore, trait-based flammability studies are an important contribution to understanding how fire shapes ecosystems and species persistence in fire-prone ecosystems \citep{pausas2012fire, pausas2017flammability}. Fuel varies structurally and many plant traits can influence flammability. \citet{schwilk2011scaling} demonstrated that it is possible to establish a link between characteristics of individual species  and fire severity in landscape level. However, although we have been witnessing an increasing trend of high-severity wildfires in many parts of the world \citep{miller2012trends, dennison2014large, weber2020spatiotemporal, salguero2020wildfire}, a mechanistic understanding of the canopy traits-flammability relationship is still in an early stage.

Plant flammability is the general capacity of vegetation to burn. Different physical and chemical characteristics of plants affect how susceptible they are to burning \citep{bond1996fire}. It is well-recognized that plants' have both inter and intra-specific differences in their flammability \citep{pausas2012fire, battersby2017exploring, cui2020shoot, cui2022intraspecific} and also differ in growth form \citep{calitz2015investigating, jaureguiberry2011device, zanzarini2022flammability, potts2022growth, cui2020shoot, wyse2016quantitative}. However, measuring the flammability of plants on an individual scale and predicting fire behavior at larger scales is challenging. One complication is that the concept of flammability is subjective \citep{gill2005flammability} and methodological differences across studies make comparisons difficult. Historically, flammability was considered as consisting of four different components: ignitibility, combustibility, consumability, and sustainability \citep{anderson1970forest, martin1993assessing}. However, empirical evidence suggests that these metrics are inter-correlated and most work indicates that flammability has three major dimensions: ignitability, total heat release, and fire spread rate \citep{pausas2017flammability}. These dimensions may be independent of one another for a particular scale but might be correlated at other scales. For example, past work in surface fire systems suggests that the flammability of leaf litter varies along two largely orthogonal dimensions, total heat release and flame spread rate \citep{scarff2006leaf, de2012leaf,cornwell2015flammability}. More recently, \citet{prior2018conceptualizing} suggested that these two main axes can be used to describe flammability at the community scale and successfully predict fire severity for a wide range of frequently measured flammability metrics.

The overall tendency of vegetation to catch and spread fire is strongly influenced by the flammability of dominant plant species. Therefore, flammability is a fundamental element in determining the ecological effects of fire \citep{bond1995kill, lavorel2002predicting, bond2005fire}. However, studies of the traits that determine a plant's flammability, particularly in individual scale, and, consequently, its unique contribution to the fire regime in a landscape, are uncommon \citep{jaureguiberry2011device, schwilk2015dimensions, pausas2017flammability}. Part of the reason is the lack of comparable, cost-effective protocols to measure the flammability of whole plants over large numbers of species \citep{jaureguiberry2011device}. Burning the whole plant would be more insightful to understanding the plant's flammability, but it is expensive and logistically challenging \citep{jaureguiberry2011device, pausasandmoi2012flammability}. Therefore, many flammability studies included burning individual leaves to predict canopy flammability and fewer considered the geometry of plant parts which has a great influence on fire behavior \citep{schwilk2003flammability, madrigal2012evaluation, gao2018grass, calitz2015investigating, pausas2012fires}.Moreover, recent studies suggest that measuring flammability by burning entire shoots is better at predicting plant flammability than burning leaves \citep{alam2020shoot} and individual leaf flammability is a poor predictor of crown fire behavior \citep{fernandes2012plant}. As a consequence, it is essential to use methods that measure the flammability of whole or partial plant canopies which retains the special arrangement of plant parts and make measurements reproducible, and can be more efficient in predicting fire patterns in the wildfire.



%%[DWS: Did you make outline of this? these paragraphs seem to rehash some of the same information and I'm not clear on exact development of the logic, AM : I made a little change in this paragraph and I guess now the development of the logic is not vague!!]


The plant traits' that influence plant flammability are different on different scales \citep{pausas2017flammability}. For example, in plants' organ level such as in leaf, the leaf's morphology and chemistry determines the way leaf burns \citep{anderson1970forest, owens1998seasonal, schwilk2011scaling, pausas2016secondary, guerrero2021leaf, ganteaume2021volatile,alam2020shoot}. While some traits, like the presence of volatile compounds and high leaf dry matter content, increase leaf flammability, others, including high moisture content, and thick leaves, can decrease flammability. In an individual scale, plant architecture such as fuel compactness, branching patterns are one of the most important factors in plant flammability \citep{schwilk2003flammability, madrigal2012evaluation}. For example, in denser canopies, the heat transfer enables the fire to readily move from one branch to another \citep{bond1996fire}. Moreover, live fuel initially acts as a heat sink, and the amount of heat needed to substantially dry the live fuel as it begins to burn varies for the leaf and stem. Therefore, the canopy's leaf cover is  also a key factor in crown fire \citep{ray2005micrometeorological}. However, the importance of canopy traits in shoot level flammability has yet to be tested \citep{alam2020shoot}.

%%[DWS: This seems to be going around in circles].  You are saying the same thing over and over again. Make an outline]

%%[DWS: Set up this dichotomy clearly. Don;t make the reader figure out the logic on their own]

Architectural branching designs vary among plants \citep{halle2012tropical} and plant architecture can be shaped by different selective pressures \citep{danell1994browseeffects, schwilk2003flammability} and interacts with other morphological and life history traits \citep{ackerly1998leaf, schwilk2001flammability,archibald2003growing}. A recent study suggests that considering the dissimilarities in branching patterns can improve the plant scaling models \citep{bentley2013empirical}. Moreover, plant shoots as phytomers are influenced by the interaction between genotype and environmental pressure \citep{mcsteen2005shoot, wang2008molecular}. Therefore, it is reasonable to treat a terminal branch from a given plant sample as a canopy trait rather than representative of the total fuel load of that plant.

%%[DWS: that is all introduction I think? OUt of place in methods, AM: I am bringing this paragraph after setting up the dicotomy, I can't see any other better place for this paragraph for now.].

My goal was to investigate the importance of traits in controlling the shoot flammability of native shrubs of Texas. %%[DWS: The introduction seems to be setting up a more interesting general question but you never get there?, AM: I believe the set up is okay now.  DWS: WHAT? did you understand my comment? My criticismm is of the sentence above. Please read wheat I write.]
I burned a total of 116 samples, 3-12 individuals per species representing 16 common shrub species of Texas for measuring flammability. I measured four common canopy traits: total biomass, canopy density, canopy moisture content, leaf: stem, and four common leaf traits: LMA, leaf area per leaflet, leaf moisture content, and leaf length per leaflet to answer two questions: 1). Are canopy traits more important than leaf traits in canopy flammability? 2). Are heat release and flame spread independent axes of flammability in shrub fuels? My research, investigating the importance of structural traits of native shrubs on shoot-level flammability might help improve the understanding of the shrub fuel-fire feedback in the shrublands of Texas.


\section{Methods}
\subsection{Site selection}\
%%[DWS: You never actually explain site selection.] You also do not explain individual selection sampling design, although that is very important]

The species were selected from the shrub species found on the properties I had access to collect samples from during my preliminary visit to those properties. Those private properties were the part of Texas Ecological Laboratory  \url{https://texasecolabprogram.org/} program which provides tax benefits to landowners who allow research on their property.  To collect samples, I traveled to fourteen different properties in eleven counties in Texas (Bandera, Edwards, Menard, Duval, Uvalde, Kendall, Bastrop, Dickens, Van Zandt, Real, Bell)(Table \ref{tab:property list}).   %%[DWS: Is there a table of sites?].
Plant samples were collected during summer (late May -- early August) 2022 from 16 native shrub species. The number of studied species from each property was maximized in light of the potential variance in flammability within species. However, only \emph{Diospyros texana}, one of Texas' most prevalent shrubs, was found at all eight properties. There are five properties where the species \emph{Juniperus ashei, Mahonia trifoliolata, Sophora secundiflora}, and \emph{Rhus virens} were found. Four properties contained three species (\emph{Senegalia wrightii, Prosopis glandulosa, and Ilex vomitoria}), while three properties contained \emph{Sarcomphalus obtusifolia} \citep{hauenschild2016phylogenetic} and \emph{Calicarpa americana}. The species \emph{Juniperus virginiana} and \emph{Forestiera pubescens} were found in two properties, but the remaining species were only found in one property each. %%[DWS: How were individuals selected?  Never explained]. 
 Due to the challenge of differentiating \emph{Senegalia wrightii} from \emph{Senegalia greggii} without flowers during sampling, I treated them as a single taxon. I recorded the coordinates of the collection site by GPS, took pictures of each sample  and collected specimens of unidentified shrubs.

\begin{table}
    \centering
    \begin{tabular}{lrr}
    \toprule
    Name of the properties & County\\
    \midrule
     Edwards 2020-22 & Edwards\\
     Kendall 2021-7    &  Kendall\\
     Kendall 2021-8   &  Kendall\\
     Kendall 2021-9   & Kendall\\
     Real 2022-13  & Real\\
     Van Zandt 2021-1 & Van Zandt\\
     Real 2022-13 & Real\\
     Dickens Spring & Dickens\\
     Duval 2022-1 & Duval\\
     Uvalde 2021-3 & Uvalde\\
     Bell 2022-09 & Bell\\
     Bandera 2021-32 & Bandera\\
     Bastrop-Fayette 2022-01 & Bastrop-Fayette\\
      Menard  & Menard\\
      \bottomrule
    \end{tabular}
    \caption{The list of properties from where the samples were collected}
    \label{tab:property list}
\end{table}





The sampled individuals were chosen from those who were in good health and did not appear to be under water stress. I collected three samples from each mature, healthy individual. Two of them comprised a pair of 70\,cm long shoot samples. These were chosen to be visually similar, healthy-looking terminal branches. I chose paired samples to allow one to be used later in flammability trials and the other to be used for destructive leaf and stem measurements. The third sample was leaves for leaf trait measurements. %%The third sample was 5--108 [DWS:5 to 108? Learn how to write dash, n-dash and m-dash in LaTeX] leaves per plant for leaf trait measurements.[AM: maximum number of leaflets, just mentioned leaves samples is okay?]
The destructive measurements shoot sample was cut into sections, sealed inside a plastic food storage bag, and placed on ice in a cooler for later measurements. The shoot sample destined for burning trials was kept intact and placed inside a large sealable plastic bag (Ziploc BIG BAGS, SC Johnson, 60 \,cm × 82 \,cm × 18 \,cm). The leaves sample was placed in a small sealable plastic bag.

A paper towel, saturated with water was placed inside the plastic bags to maintain 100 \% relative humidity inside the bag and eliminate further drying after cutting. To avoid puncturing the plastic bag, I used a large cotton towel to wrap samples from species with spines, needles, or thorns. All the samples were placed inside an insulated cooler on ice and transported to Texas Tech University within five days of collection.  %%[DWS: how many?]. 
At the lab, the intact shoot samples were left to bench dry for 36 hours in the lab before burning trials  as per established method \citep{wyse2016quantitative}. %%to mimic combustion properties under drought conditions [DWS: either justify fully or don't go into detail at all
A data logger (HOBO MX2300 temp/RH, Onset, Bourne, MA) recorded the temperature and humidity in the lab during the drying period at one second intervals.

%\url{https://microdaq.com/onset-hobo-outdoor-bluetooth-humidity-data logger.php}) %%[DWS: Look up how one cites manufacturers. a url is not correct


\subsection{Trait measurements}
The sampled branch assigned to destructive harvests was used for measuring the leaf and stem biomass.
%[DWS: You had some weird typos here with words omitted?].
This branch was separated into leaves and stems for separate drying and weighing. The leaves, including petiole and rachis, were separated from the stem and their fresh mass was measured before placing them in a paper bag for drying. Following the measurement of the fresh mass, the leaves and stems were each dried at $65^{\circ}$C for at least 48 hours before weighing.

Canopy density was measured during the drying period of the burned sample. I visually assigned each branch as approximating one of three possible geometric shapes: cylinder, truncated cone, or the combination of two truncated cones connected at their wide end. For those samples assigned the cylinder shape, I measured the diameter of the branch at three different positions, averaged those measurements, and calculated the volume according to the formula of the volume of a cylinder. For those samples assigned the truncated cone shape, I measured the diameter at both ends and calculated it according to the appropriate equations. For those samples assigned the double truncated cone shape, I measured the maximum diameter, the distance of the maximum diameter from the distal and proximal ends of the branch, and the diameters at the distal and proximal ends. I then calculated the combined volume of the truncated cones. Canopy density was calculated by dividing the total dry mass of the sample by the canopy volume.

The leaflet of compound leaves is functionally similar to a simple leaf \citep{perez2016corrigendum}. Therefore, I used the leaflet for compound leaves to measure the leaf traits. Leaf area per leaflet, leaf length per leaflet, and leaf mass per area (LMA) were measured on the leaf samples. I used five leaflet for the majority of the species. For species with small leaflets the number of leaflet was not fixed.
%%[DWS: What do you mean? I assume you still used five full leaves for compound leaves? LMA of a compound leaf is the whole leaf, right? Overall, how you dealt with compound leaves is unclear]. 
I used a ruler scale to measure each leaflet's length, including the petiole, and then I weighed the leaves both when they were fresh and then after 48 hours of 65 $^{\circ}$C oven drying. A Licor 3100 leaf area meter, which can measure the surface area of leaves, was used to measure the total fresh leaflet area. Each time before measurements, the leaf area meter was calibrated. The leaflet was positioned so that it lies flat on the bed and does not overlap any other leaflets.   %%[DWS: full information?].
LMA was then calculated on  dry mass basis. The projected and total leaf area are different for non-flat leaves \citep{perez2016corrigendum, cornelissen2003handbook}. Therefore, I measured the leaf area of three \emph{Juniperus} species manually. I assumed that the scale-like leaves, which were stayed apart from one another in the samples under study, had a cylindrical shape. The length and width of the leave were treated as the height and diameter of a cylinder respectively. Ten leaves for each sample were used and I measured the diameter and length of each leaf with slide calipers and a meter scale respectively. Based on the formula for calculating a cylinder's surface area, the area of the leaves was calculated. The 10 leaves were weighed fresh and then dried in an oven for 48 hours at 65 $^{\circ}$C to obtain the dry weight.

%[DWS: So why not branchlet dimension s like in other literature?  I'm confused on this. You treat Cupressaceae differently than the litter flammability studies have. Can you explain why?]

\section{Flammability trials}

I conducted flammability trials using a specially-built device that allows standardized measurement of the canopy flammability of portions up to 70\,cm long \citep{jaureguiberry2011device}. The chosen procedure calls for preheating the 70\,cm branch on the grill (84 x 55 \,cm) for two minutes before lighting the sample with a blowtorch. The grill was heated by a flame created by propane gas and preheating is crucial for simulating wildfires in nature, because live fuel first acts as a heat sink after being exposed to tremendous heat before drying out and eventually reaching a flash point when they begin to burn. In contrast to the adopted procedures, where they used a temperature gauge connected to the grill thermometer to monitor the grill temperature, I used two black aluminum discs  suspended from the wind protection shield and spaced approximately 19 \,cm (the center of the discs) apart from the grill's surface. This method is similar to that employed by \citet{gao2022burn} to measure heat transfer. Before placing the samples on the grill, the temperature of the discs was measured and the heat output of the propane burner was adjusted to keep these disks within a ranged from 130.25 to 163.30 $^{\circ}$C, with a mean of 148.79 $^{\circ}$C temperature  during the experiment. The two minute preheating period before ignition is intended to equalize air temperature during the burn trial, however wind speed could influence burning and I used
a Kestrel 3000 pocket weather meter (Nielsen–Kellerman, Co.,Chester, PA, USA) to
measure the maximum wind speed before each trial. 
%%[DWS: explain this]. 
The pre and post-burning temperatures of each trial were also recorded to quantify the heat release (J). I also measured the flame height using a meter scale that was mounted to the flammability apparatus.

%[DWS: Oh, I dont like this and should not ahve been required. Too many covariates, I think. Was this necessary. Dang. I thought you adjusted heat output on burner]. I used a Kestrel 3000 pocket weather meter (Nielsen–Kellerman, Co.,Chester, PA, USA) to measure the maximum wind speed before each trial.

A few leaves (range: 0.044 -- 2.380 g, mean: 0.408 g) and a twig with leaves (range: 0.0796 -- 8.177 g, mean: 1.11 g) were detached from the sample to be burned right before burning in order to calculate the leaf moisture content and canopy moisture content respectively. The leaves and twig with leaves were weighed fresh and then dried in an oven for 48 hours at 65 $^{\circ}$C to obtain the dry weight and calculated the leaf moisture content and canopy moisture content in dry mass basis.
%%after it had been air-dried, and  [DWS: VAGUE. CLarify this as I've asked before].
I measured the fresh weight of each sample using a portable balance (0.1 g readability) right before burning. %%[DWS: accurate to ?]. 
I then laid the sample horizontally on the grill for two minutes for preheating. A Fluke 572-2 IR remote infrared thermometer, which can measure temperatures up to 900 $^{\circ}$C, was used to measure the aluminum discs' precombustion temperatures during the pre-heating phase. Some preliminary flammability trials %%in 2021 [DWS: first mention? data citation?] 
showed that some samples from the most flammable species didn't ignite within the allotted 10 seconds ignition period, therefore, I kept the blow torch on until a sample caught fire and treated the ignitibility as ignition delay \citep{anderson1970forest}.  
%%[DWS: is this necessary to explain? Do you report 2021 data? AM: Is it okay now?]

I used a thermocouple recording system to keep track of the flame temperatures during the burning trials. The flame temperatures were continually collected at three sites at intervals of one second using the most widely used, K-type thermocouple sensors \citep{mcgranahan2020inconvenient} attached to data loggers. %%([DWS: cite]).
An assistant continuously monitored the highest flame height for each trial and measured it using the meter scale. When the flame was extinguished, I measured the post-burn temperature of the discs, the time it took to ignite, and the overall burning time in seconds with a stopwatch. After the burn, I calculated the percentage consumed and weighed the leftover biomass. To reduce observer error, the proportion of the volume burned was visually estimated and agreed upon by at least two people. Those samples that caught fire during preheating were counted as zero seconds as ignition delay. Each trial's average gas flow from the propane gas cylinder was 20.35\,$g min^{-1}$. Three flammability metrics: temperature integration over 100$^{\circ}$C, duration over 100$^{\circ}$C, and peak temperature were measured using thermocouples data.

\section{Data analysis}

All data analysis was conducted in R, version 4.2.2 \citep{R}. 
%% [DWS: you had wrong citation for that version]. 
Flammability includes separate measurements that do not necessarily correlate with one another, therefore the first step before further analysis was to use principle components analyses (PCA) to transform the measured variables into orthogonal axes of variation. The measured variables were in different units. Therefore, correlation matrix in the \pkg{prcomp} function was specified. In terms of fire ecology, it is not just the amount of temperature generated during fire that is important, but also the duration \citep{mcgranahan2020inconvenient}, because the thresholds for temperature and time-related mortality driven by fire varied for different organisms \citep{nelson1952observations,vines1968heat, bond1983dead, hengst1994bark,pinard1997fire,lawes2011bark, pingree2019myth} due to the avoidance–attraction traits for fire \citep{schwilk2001flammability, archibald2019unified}. Moreover, seasonal variations also affect how long a plant must be exposed to a certain temperature before killed by fire \citep{wright1970method}. As a result, the response variable temperature integration ($^{\circ}$C.s), which represented the sum of the average temperatures over each second during the same period, often use as a proxy measurement of heat release, was chosen \citep{gao2018grass, mcgranahan2020inconvenient}. Moreover, the effect of canopy traits on ignitibility has not been tested yet for shoot level flammability. Therefore, the ignition delay was also selected as a response variable.

Due to the hierarchical structure of the data, for determining the importance of traits on flammability, I built linear mixed effects models with genus as a random variable since flammability is phylogenetically conserved \citep{cui2020shoot} and I assumed that morphologically similar species would behave similarly in a wildfire. However, two species from the genus \emph{Rhus}: \emph{Rhus virens} and \emph{Rhus trilobata}, and two species from the genus \emph{Senegalia}: \emph{Senegalia wrightii} and \emph{Senegalia berlandieri} are morphologically distinct. Therefore, they were treated as a single taxon. 



%%[DWS: You did some grouping in the analysis by taxonopmy but that is never explained?]

To avoid the  Type I error and an overestimation of the amount of variance explained that could arise from forward selection method \citep{blanchet2008forward}, initially, a global mixed-effect model was built for each type of trait for temperature integration with two-way interaction to select the best predictors for each type of traits.
%[DWS: This makes no sense. Why temperatuere integration after you measured so much? What are ``types'' of traits?  Nothing is explained].
The \pkg{lmer} function from the afex package \citep{singmann2015packageafex,afexluke2017evaluating} was used to perform the mixed-effect model.
%[DWS: Why mixed effects models. The methods don;t anticipate this]. 
To check the collinearity among predictor variables and make the fixed effects comparable, I created the correlation matrix and standardized the independent variables as z-score. Furthermore, after checking the residuals of the models, the temperature integration over 100 $^{\circ}$C was log-transformed. 
%[DWS: but figures suggest otherwise?]. 
The Kendall rank correlation coefficient between leaf area per leaflet and leaf length per leaflet is 0.65 and I decided to use the leaf length per leaflet in the model because previous flammability studies used leaf length per leaflet to predict flammability \citep{alam2020shoot}. 
%[DWS: So dropping post-hoc? THis sounds like a mess]. 
The best model from all the subsets of the global model for each type of trait was selected by The Akaike information criterion ($AIC_{c}$).

%[DWS: this whole approach needs to be clearly laid out! It is confusing and sounds like a fishing expedition although it is not]
The automated model selection was performed by the \pkg{MuMin} package \citep{barton2015packagemumin}. I compared the best leaf traits model against the best canopy traits model according to log-likelihoods. After selecting the best predictors for each type of traits, the mixed effect models with best leaf and canopy traits were built with the \pkg{lme4} package in R \citep{bates2009package} using maximum likelihood estimation in order to evaluate the significance of fixed effects. I used the Anova() function from the \pkg{car} package \citep{fox2013hypothesis} to test the significance of predictors. To avoid unacceptable type 1 error, estimated degrees of freedom of residuals and p-values were calculated using the Kenward-Roger approximation \citep{kenward1997small}. All the plots were generated by \pkg{ggplot2}, \pkg{factoextra}, and \pkg{ggpubr} package \citep{wickham2016packageggplot2, kassambara2017packagefactoextra,kassambara2020package}.

%The two-minute preheating period before ignition is intended to equalize air temperature during the burn trial, however, wind speed could influence burning. Therefore, a linear regression was performed to test the effect of wind speed on flammability. 



%[DWS: Set up the approach before talking about the details. This MUST be clear or this will never make it through review and will be rejected outright].

\section{Results}

\subsection{Flammability axes}
According to the principle component analysis, the first two principal components accounted for 77.8\% of the variation of all the measured flammability metrics. The first axis %% which represents heat release [DWS: How do you know what it ``represents''?], 
captured 68.5\% of the total variation (Figure \ref{fig:pca-plot}). The temperature integration in the first axis received the most contributions and six flammability traits had a strong positive correlation with one another (loadings: temperature integration over 100 $^{\circ}$C = 0.38, flame duration = 0.38, duration over 100C $^{\circ}$C = 0.37, mass-consumed = 0.37, peak temperature  = 0.36, percentage of volume burned = 0.35). The second axis of PCA explained 9.36 \% of the total variation and the ignition delay and heat release (j) both contributed the most in the second principle component and are negatively correlated with each other (loading: ignition delay = -0.75, heat release (j) = 0.58). 
%%[DWS: Table? any figures? You do not cite any figures!]

\begin{figure}  % DWS: You never cite this so why include it?
    \centering
    \includegraphics[width = \textwidth]{../../results/pca_plot.pdf}
    \caption[Principle components results]{\label{fig:pca-plot} Principle component analysis biplot of nine flammability traits with their abbreviations (ID = Ignition delay, FD = Flame duration, TI = Temperature integration over 100 $^{\circ}$C, Duration over 100 $^{\circ}$C, MC = Mass consumed, VB = Volume burned, PT = Peak temperature, FH = Flame height, HR = Heat release (j)) with their quality of representation as $cos^2$ (squared coordinates). A high $cos^2$ indicates a good representation of the variable in the principle component and a low $cos^2$ of a variable indicates less importance in the principle component}
  \end{figure}

%%[DWS: You have figures you include but never cite. I added exmaple citations here but I can't figure out where the code produces this figure. ]

\subsection{Effect of canopy traits and leaf traits on temperature integration}
%%To examine the impact of wind speed on flammability [DWS: why is this interesting? Why are you doing this this was not a question you introduced and seems out of place], a linear regression was first conducted [DWS: NO, why? This is bad]. The results showed that wind speed had no effect on the temperature integration over 100 $^{\circ}$C (p > 0.05). [DWS: But how can linear regression answer this in a multivariate relationship? You are doing dangerous things here].

The automated model selection result shows that, for canopy traits, the model with total dry mass and canopy density without interaction (Table \ref{tab:canopy_models}) is the best fit to predict temperature integration and \MakeUppercase{lma} alone is the best fit as predictor among all the leaf traits (Table \ref{tab:leaf_models}). The model with canopy traits is better at fitting the data set compared to leaf traits ($AIC_{c}$ value: best canopy traits model = 3.18, best leaf traits model = 76.26). I built another linear mixed effect model using the best canopy traits, mean pre-burning disc temperature and wind speed as a covariate to see whether adding pre-burning temperature and wind speed during trials improves the model or not. I found that, adding pre-burning temperature and wind speed didn't improve the model ($AIC_{c}$ value: best canopy traits model = 3.18, pre-burning temperature model = 5.73).

\begin{table}
  \centering
  \caption{Model selection results for top four models from the global linear
    mixed effect model for canopy traits for temperature integration with
    corresponding $AIC_{c}$ value and model weight. The best model is highlighted in bold}
  %\vspace{0.5 cm}
  \begin{tabular}{lrrr}
    \toprule
    \textbf{Fixed terms} & $AIC_{c}$ & \textbf{Weight}\\
    \midrule
    \textbf{total dry mass + canopy den.}    & \textbf{3.7} &  \textbf{0.324}\\
    total dry mass + canopy den. + total dry mass:canopy den. & 5.4  & 0.142 \\
    total dry mass + moisture content + canopy den.   & 5.4   & 0.140 \\ 
    total dry mass + leaf ratio + canopy den.  & 5.4 & 0.138  \\
    \bottomrule
  \end{tabular}
  \label{tab:canopy_models}
\end{table}

%% DWS: ":" in a model specification implies interaction term. You need to call
%% your leaf:stem ratio something like leaf ratio here,

%[DWS: NO tables or figures of results?]

\begin{table}
  \centering
  \caption{Model selection results for top four models from the global linear
    mixed effect model for leaf traits for temperature integration with
    corresponding $AIC_{c}$ value and model weight. The best model is highlighted in bold}
  %\vspace{0.5 cm}
  \begin{tabular}{lrr}
    \toprule
    \textbf{Fixed terms} & $AIC_{c}$ & \textbf{Weight}\\
    \midrule
    \textbf{LMA} & \textbf{76.6} & \textbf{0.326} \\
    LMA + leaf length per leaflet + LMA:leaf length per leaflet & 78.2     & 0.148    \\
    LMA + leaf moisture content                                 & 78.3     & 0.142    \\
    LMA + leaf length per leaflet                               & 78.4     & 0.134    \\
    \bottomrule
  \end{tabular}
  \label{tab:leaf_models}
\end{table}



%[DWS: This table only shows two models, but the model selection was on a next set from a full models. Then you did a two model test, right?]

After getting the best predictors for temperature integration for each type of traits from the initial model selection,
%[DWS: ??? best?],
I tested the significance of the predictors. Total dry mass and canopy density had a significant positive linear effect %[DWS: But I thought variable was lof-scaled?  What do you mean?] 
on temperature integration above 100 $^{\circ}$C (Figure \ref{fig:dm-tempint}). Marginal $R^2$ for the model with total dry mass and canopy density without interaction was 0.51 (total dry mass: p $<$ 0.001, canopy density: $p = 0.026$) (Table \ref{tab:fandpstatfortemp}) (Figure \ref{fig:canopyden-tempint}). LMA had also a significant positive effect on temperature integration above 100 $^{\circ}$C. Marginal $R^2$ for the model with LMA was 0.040 (LMA: $p = 0.032$). 

The majority of the canopy traits among all the measured canopy traits have higher values for the species from the \emph{Juniperus} taxon, especially \emph{Juniperus ashei} and \emph{Juniperus pinchotii} and the measurements of the leaf traits is also error prone due to having the non--flat leaves for those species. In order to understand how this taxon influenced the results, I removed the species from the \emph{Juniperus} taxon from the analysis and then examined the impact of those traits on the remaining species. After removing the \emph{Juniperus} species from the analysis, the total dry mass still had a strong positive effect ($p $<$ 0.001$), but canopy density had a marginal effect on temperature integration ($p = 0.073$). However, LMA didn't have any significant effect on temperature integration for the remaining species ($p > 0.5$) 

% I will report this result on supplimentary info.

%[DWS: not how to report this. Check with me].

\begin{table}
\centering
\caption{Mixed effect model coefficients and ANOVA table for the best predictors for each type of traits for temperature integration over 100 $^{\circ}$C. The linear mixed model was fitted with lmer() function from \pkg{lme4} package \citep{bates2009package}. The \pkg{car} package in R \citep{fox2013hypothesis} was used to calculate the estimated degrees of freedom of residuals, F and p-values using the Kenward-Roger approximation \citep{kenward1997small}. All the independent variables were standardized  as z-score and the response variable was log transformed.}
\vspace{0.5 cm}
\begin{tabular}{lrrrr}
  \hline
 &  Estimate & F  & Df.res & Pr($>$F) \\ 
  \hline 
  total dry mass (g) & 0.2587 & 72.4607  & 107.6540 & \textbf{$<$0.001} \\ 
  canopy density (g/{$cm^3$}) & 0.0583 & 5.0558  & 103.1509 & \textbf{0.026} \\ 
  LMA & 0.0922 & 4.6712 &  105.1887 & \textbf{0.032} \\ 
   \hline
\end{tabular}
\label{tab:fandpstatfortemp}
\end{table}

%[DWS: Why not a model table?]

\begin{figure}
    \centering
    \includegraphics[width = \textwidth]{../../results/total_dry_mass.pdf}
    \caption[Dry mass effect on temperature integration]{\label{fig:dm-tempint}Relationship between total dry mass and temperature integration above 100 $^{\circ}$C. The line indicates the best-fitted linear mixed model with species as a random intercept effect ($p $<$ 0.001$). Small points in the background are individual observations, and large points are species means. The red dots are \emph{Juniperus} taxon}
\end{figure}


\begin{figure}
    \centering
    \includegraphics[width = \textwidth]{../../results/canopy_density.pdf}
    \caption[short version]{\label{fig:canopyden-tempint} Relationship between canopy density and temperature integration above 100 $^{\circ}$C. The line indicates the best-fitted linear mixed model with species as a random intercept effect ($p = 0.026$). Small points in the background are individual observations, and large points are species means. The red dots are \emph{Juniperus} taxon}
\end{figure}

\begin{figure}
    \centering
    \includegraphics[width = \textwidth]{../../results/LMA.pdf}
    \caption[short version]{\label{fig:LMA-tempint} Relationship between LMA and temperature integration above 100 $^{\circ}$C. The line indicates the best-fitted linear mixed model with species as a random intercept effect ($p = 0.032$). Small points in the background are individual observations, and large points are species means. The red dots are \emph{Juniperus} taxon.}
\end{figure}


%[DWS: Three figure you never use? Why? Why three univariate relationships and not figures that better represent the models? Do all of the results basically just show juniperus as biggest effect? I see models without Juniperus in the analysis code.]


\subsection{Effect of canopy and leaf traits on ignition delay}

%## DWS: why ignition delay? This is all so confusing. You have chosen two traits out of the many you looked at but never explain why. You never explain the approach. Is this based on theory? past work, the PCA? ]

%For ignition delay, all the models were fitted without any interaction term to avoid the overfitting problem [DWS:  ??? So why is it a problem here but not elsewhere? These decisions and explanation are confusing and any reviewer will reject this].

For ignition delay, the global linear mixed effect model was built without any interaction term  to avoid over fitting problem since ignition source influences ignitibility \citep{madrigal2012evaluation} and many unmeasured traits like volatile compounds can influence ignitibility. For canopy traits, the model with canopy density and canopy moisture content is the best fit to predict ignition delay (Table \ref{tab:canopy_models_ignition_delay}) and LMA and leaf moisture content is the best fit as predictor among all the leaf traits (Table \ref{tab:leaf_models_ignition_delay}). The model with canopy traits is better at fitting the data set compared to leaf traits ($AIC_{c}$ value: best canopy traits model = 579.15, best leaf traits model = 584.87). I built another linear mixed effect model using the best canopy traits,  mean pre-burning disc temperature and wind speed as a covariate to see whether adding pre-burning temperature and wind speed improves the model or not. I found that, adding pre-burning temperature and wind speed didn't improve the model ($AIC_{c}$ value: best canopy traits model = 579.15, pre-burning temperature model = 582.2).

%[DWS: I thought we came up with a simple approach. You never explain it]


\begin{table}
  \centering
  \caption{Model selection results for top four models from the global linear
    mixed effect model for canopy traits for ignition delay with corresponding
    $AIC_{c}$ value and model weight. The best model is highlighted in bold.}
  % \vspace{0.5 cm}
  \begin{tabular}{lrrr}
    \toprule
    \textbf{Fixed terms} & $AIC_{c}$ & \textbf{Weight}\\
    \midrule
    %% DWS; still does not fit. Need to fix or make a landscape table
    \textbf{canopy den. + moisture content}    & \textbf{579.7} & \textbf{0.425} \\
    canopy den. + moisture content + total dry mass   & 580.6   & 0.275 \\
    canopy den. + moisture content + leaf ratio       & 581.6   & 0.166  \\
    canopy den. + moisture content + leaf ratio + total dry mass & 582.7  & 0.095 \\
    \bottomrule
  \end{tabular}
  \label{tab:canopy_models_ignition_delay}
\end{table}


\begin{table}
  \centering
  \caption{Model selection results for top four models from the global linear
    mixed effect model for leaf traits for ignition delay with corresponding
    $AIC_{c}$ value and model weight. The best model is highlighted in bold.}
  \begin{tabular}{lrrr}
    \toprule
    \textbf{Fixed terms} & $AIC_{c}$ & \textbf{Weight}\\
    \midrule
    \textbf{LMA + leaf moisture content}    & \textbf{585.4} & \textbf{0.596} \\
    Leaf length per leaflet + LMA + leaf moisture content & 586.9          & 0.288          \\
    LMA                                                   & 589.8          & 0.067          \\
    Leaf length per leaflet + LMA                         & 591.9          & 0.023          \\
    \bottomrule
  \end{tabular}
  \label{tab:leaf_models_ignition_delay}
\end{table}



%[DWS: another figure you never use?]


%After selecting the best models, I checked how well the models fit the data set.  [DWS: this is repetetive. Just refer tot he model tables (which I cannot find?]].  
After testing the significance of the predictors on ignition delay, I found that all the predictors from the best canopy (Figures \ref{fig:canopy_density_ig_delay} and \ref{fig:canopy_moisture_ig_delay}) and best leaf traits (Figures \ref{fig:lma_ig_delay} and \ref{fig:leafmc_ig_delay}) model had a strong positive effect on ignition delay. The marginal $R^2$ for the model with canopy density and canopy moisture content was 0.21 (canopy density: $p = 0.003$, canopy moisture content: $p = 0.006$) , and the marginal $R^2$ for the model with LMA and leaf moisture content was 0.15 (LMA: $p = 0.004$, leaf moisture content: $p = 0.010$). (Table \ref{tab:fandpforig_delay})

\begin{table}[ht]
  \centering
  \caption{Mixed effect model coefficients and ANOVA table for the best
    predictors among the measured canopy and leaf traits for ignition delay.
    The linear mixed model was fitted with lmer() function from \pkg{lme4}
    package \citep{bates2009package}. The \pkg{car} package
    \citep{fox2013hypothesis} was used to calculate the estimated degrees of
    freedom of residuals, F and p-values using the Kenward-Roger approximation
    \citep{kenward1997small}. All the independent variables were standardized
    as z-scores.}
  % \vspace{0.2cm}
  \begin{tabular}{lrrrr}
    \toprule
    & Estimate & F & Df.res & Pr($>$F) \\ 
    \midrule
    canopy density (g/{$cm^3$}) & 0.9694 & 8.7956  & 110.3028 & \textbf{0.003} \\ 
    canopy moisture content (\%) & 0.9103 & 7.8908 & 59.4044 & \textbf{0.006} \\ 
    LMA & 1.0053 & 9.0544  & 54.6628 & \textbf{0.004} \\  
    leaf moisture content (\%) & 0.9208 & 7.0012  & 73.3568 & \textbf{0.010} \\ 
    \bottomrule
  \end{tabular}
  \label{tab:fandpforig_delay}
\end{table}

After removing \emph{Juniperus} from the analysis, except leaf moisture content, none of the traits from the best models had significant effect on ignition delay (canopy density: $p = 0.423$ ,
LMA: $p = 0.330$, leaf moisture content: $p = 0.048$). Leaf is one of the first plant parts to catch fire.
Therefore, since leaf moisture content and canopy moisture content are highly correlated, the canopy moisture content was removed from the model that was used to test the effect of those traits after removing species from the \emph{Juniperus} taxon.



%[DWS: I thought this was a hypothesis testing paper and it has devolved into variable selection].



\begin{figure}
    \centering
    \includegraphics[width = \textwidth]{../../results/canopy_density_ignition.pdf}
    \caption{Relationship between canopy density and ignition delay. The line indicates the best-fitted linear mixed model with species as a random intercept effect ($p  = 0.003$). Small points in the background are individual observations, and large points are species means. The red dots are \emph{Juniperus} taxon}
    \label{fig:canopy_density_ig_delay}
\end{figure}

\begin{figure}
    \centering
    \includegraphics[width = \textwidth]{../../results/canopy_moisture_ignition.pdf}
    \caption{Relationship between canopy moisture content and ignition delay. The line indicates the best-fitted linear mixed model with species as random intercept effect ($p = 0.006$). Small points in the background are individual observations, and large points are species means. The red dots are \emph{Juniperus} taxon}
    \label{fig:canopy_moisture_ig_delay}
\end{figure}

\begin{figure}
  \centering \includegraphics[width =
  \textwidth]{../../results/LMA_ignition.pdf}
  \caption{Relationship between LMA and ignition delay. The line indicates the best-fitted linear mixed model with species as a random intercept effect ($p = 0.004$). Small points in the background are individual observations, and large points are species means. The red dots are \emph{Juniperus} taxon}
  \label{fig:lma_ig_delay}
\end{figure}



\begin{figure}
  \centering \includegraphics[width =
  \textwidth]{../../results/leaf_moisture_ignition.pdf}
  \caption{Relationship between leaf moisture content and ignition delay. The line indicates the best-fitted linear mixed model with species as random intercept effect ($p = 0.010$). Small points in the background are individual observations, and large points are species means. The red dots are \emph{Juniperus} taxon}
  \label{fig:leafmc_ig_delay}
\end{figure}


%[DWS: Many more figures you never use. What is the point of all this? What is the story?]


\section{Discussion}

I found that canopy traits are more important than are leaf traits in shoot flammability in shrub fuels. %[DWS: great, I thought this was the main question but the methods and results don't match. Is this all driven by juniper or not?]
Specifically, species with higher amounts of fuel per 70\,cm length and higher canopy density burned for longer period of time and produced more heat during burning . These observations are consistent with the importance of the spatial arrangement of plant parts such as the bulk density \citep{pausas2012fires} and twigginess \citep{potts2022growth} in small--scale flammability experiments. Moreover, this observations are also consistent with the importance of canopy architecture and fuel compactness  in fire behavior in individual scale \citep{schwilk2003flammability, madrigal2012evaluation} which demonstrate the potential of scaling up the fire behavior by burning shoot.

%[DWS: sentence?]. % [DWS: learn how to refer to ratios, this was a mess throughout] [DWS: I do not understand this. You are talking about higher moments of the distribution or some time-scale effect? differences in variaton?].  [DWS: Talk to me about how to dsicuss this. This is unexplained variance. I think you are saying that it is higher for some traits and therefore you have less power to detect effects with some measurements?]. 
%[DWS: Why does the whole first paragraph focus on a negative result about leaf:stem ratio? What is theory there? I did not think it was a major hypothesis of yours but this negative result is getting top billing in the thesis?]

%[DWS: OPh, ther eare model tables, great! Why did you never refer to them?]

%[DWS: You don't use this table? It is never cited]


Leaf mass per area (LMA), one of the most important traits from the Leaf Economic Spectrum \citep{wright2004worldwide} is appeared to be the most important determinants of temperature integration among all the measured leaf traits in this study. LMA is related to leaf thickness \citep{niinemets1999research}, and species with high LMA generally have more dense leaf tissue \citep{poorter2009causes}. Like leaf dry matter content (LDMC) in shoot flammability \citep{alam2020shoot, potts2022growth}, this observation
was also likely due to the denser leaf tissue, which allows for longer burning with greater intensity. This result is consistent with previous leaf flammability experiments \citep{krix2018landscape}.
However, the significance of this relationship doesn’t hold true after removing the most flammable taxon from the analysis, which further confirms that the temperature integration is mostly driven by canopy traits in this experiment.\\

I found the two most important determinants of ignition for canopy traits were canopy density and canopy moisture content.
% canopy moisture content and leaf moisture content are highly correlated, therefore trating them as a single variable as fuel moisture content is ok?
The more compact the litter bed, the lesser the oxygen  \citep{scarff2006leaf, van2012species, engber2012patterns, de2012leaf,cornwell2015flammability}. %[DWS: What? Do you think there is oxygen limitation here?  Maybe for juniper. Any citations? That goes against most assumptions and my past papers].. 
Therefore, unlike litter flammability where the overall flammability is oxygen-limited \citep{schwilk2015dimensions}, in shoot-level flammability, to some extent, only the ignition of fuel might be oxygen limited. %DWS: ok, good idea. Back up and cite].
For the leaf traits, the thicker leaves took longer to ignite compared to the thinner leaves which is consistent with previous shoot flammability study \citep{alam2020shoot}.  %[DWS: understand the species vs other taxon names] 
Unlike some leaf flammability studies where they found that the thicker leaves burn slowly and produce less heat during burning, in shoot flammability, some species may take longer to ignite but produce more heat during burning. Moreover, \citet{madrigal2012evaluation} also suggested that the ignition source also influences ignitibility. Therefore, for cases like this, for a constant heat for the source of ignition, particularly if the heat source produces less heat than exothermic reaction, quantifying the ignitibility as ignition delay as suggested by \citep{anderson1970forest} might be appropriate to capture this phenomenon in shoot level flammability experiment.

Aside from moisture content, % since canopy moisture content and leaf moisture content are highly correlated, is okay to say just fuel moisture content? 
none of the traits from the best models significantly affected ignition delay after the most flammable taxon \emph{Juniperus} was excluded from the analysis. %[DWS: This worries me ebcause you used a random effects model so it should not be so sensitive to one species]. 
This finding implies that, ignition delay is mostly driven by moisture content in this experiment. Previous study suggested that the ignition delay in shrub fuel is less than 15 seconds when the live fuel moisture content of leaves %[Did you report fuel moisture content? This is important for comparison. Be clear] 
is less than 100\% but some species from \emph{Juniperus} taxon take longer to ignite which is consistent with this study \citep{dimitrakopoulos2001flammability,pellizzaro2007seasonal}. The maximum leaf moisture content of burning samples
%[same as fuel mc?] 
in this study was 108.05 \% in \emph{Forestiera pubescens} but the maximum ignition delay is 18 seconds for species from \emph{Juniperus} taxon.
Moreover, some species from this study are well known for the volatile components in their leaves and some of them  ignited during the pre-heating period. Therefore, the ignitibility in shoot flammability might be sometimes difficult to determine by measuring only leaf functional and structural traits.
% [DWS: ok, yes, chemistry is ignored here]

% [DWS: ????].
Canopy density influences airflow in the canopy \citep{cionco1978analysis} and might be less effective for a single shoot to understand the impact of the limitation of the airflow and how a large amount of biomass influences the fire spread rate. %[DWS: sentence does not make sense: influential to understand?.
Therefore, future study is required to investigate to what extent the geometry of a single shoot can capture the structural variation in three-dimensional space in terms of fire behavior. Moreover, in some ecosystems where the forest floor is wet, the fire spread rate in the canopy is high-density limited  \citep{ray2005micrometeorological}. Therefore, although the leaf traits like leaf length per leaflet are less important in shoot-level flammability, in some ecosystems they might be influential in crown fire since species with larger leaves might be more aerated in the canopy, which affects the rate at which fires spread by permitting the flow of the hot, dry air. 

Texas and fire have a long history together \citep{moir1982firehistory, stambaugh2011firehistory,stambaugh2014historicalfirehistory,smeins2005historyoffire1}, and there are fire adapted species in this study (e.g. \emph{Prosopis glandulosa}) \citep{glandulosahoney,wright1976effect} %[DWS: hm. debatable!]. 
Moreover, all of the most flammable shrubs are cone and seed bearing plants in this experiment 
%[DWS: what do you mean? they had fruits or cones on when burned? Or that they are simple seed plants and no ferns?] 
and study suggests that fire influences seed coat chemistry \citep{mcinnes2022doesseedcoatchemistry} and in general, most of the flammable plants are reseeders \citep{midgley2011pushingreseeders}. Therefore,  the observed strong positive effect of total dry mass per 70\,cm length and shoot density on flammability might be selected to spread the fire to the surrounding plants because if the subsequent fires kill nearby, less flammable neighbors and also increases fecundity, flammability may improve inclusive fitness \citep{bond1995kill}. However, those structural traits might have other competitive advantages such as light harvesting. Moreover, Texas has a long history with grazing as well and herbivores can also influence the architecture of plants \citep{danell1994browseeffects}. Previous study regarding \emph{Juniperus ashei}, one of the most flammable species in this study, suggested that the enhanced flammability might be the response to herbivores \citep{owens1998seasonal}. Moreover, a recent study also shows that \emph{Juniperus ashei} has an allelopathic relationship with \emph{Bouteloua curtipendula}, a native grass in this ecosystem \citep{young2009assessmentallelopathy}. Therefore, future study is required to investigate whether it is a single evolutionary force or a combination of different selective pressure that drove the enhanced flammability in these ecosystem.

\chapter{Exploring the relationship between plant defense strategies, white-tailed deer preference, and flammability in native shrub species in Texas} 

\section{Abstract}

Grazing and fire have a long history in Texas. Fire and herbivores can both increase plant flammability and alter the composition of biomes. Flammability and the extent of herbivores' ingestion are influenced by plant functional traits. A plant can defend itself from herbivores' that browse by using physical defense, chemical defense, or a combination of the two and can persist in fire-prone ecosystem through their different life history strategies. In light of the recent advancement in our understanding of the trade-off among different types of plant defense against herbivores and the relationship between plant flammability and palatability, I tested the difference in flammability between armed and unarmed species as well as two groups of shrubs based on white-tailed deer preference to answer two questions: 1) Does least preferred shrubs for white-tailed deer are more flammable than moderate to low preferred shrubs? 2) Do armed plants have lower flammability than unarmed plants? I found that the least preferred shrubs are more flammable than moderate to low preferred shrubs. However, I haven't found any significant difference in flammability between
unarmed and armed species. This study might help to improve the understanding of the trade-off between the physical and chemical defense of plants against herbivores and the unified framework for fire and herbivores' effects on plant life history. 


\section{Introduction}

Both fire and herbivores shape community assembly by acting as an ecological filter \citep{belsky1992effects, grazingecologicalfilters, morphospace, verdu2007ecologicalfilter,fireecologicalfitlers}. While mammals that eat plants primarily consume leaves and succulent twigs, fire, in many cases, can consume the majority of the above-ground plant biomass \citep{bond1996fire,globalherbivore}. Though the nature of fire and herbivores is different as a process where the fire is more abiotic while the herbivore is more biotic  \citep{globalherbivore,archibald2019unified}, some ecosystems can be shaped by both \citep{van2003effects,  archibald2005shaping,staver2009browsing,donaldson2018ecological, noy1995interactive}, one's presence affect another\citep{holdo2009grazers, foster2015synergistic} and both can enhance flammability \citep{white1994monoterpenes, owens1998seasonal, Ulex}. Plant functional traits influence flammability and the intensity of the consumption by browsing animals. However, the fire-vegetation feedback and herbivore-vegetation feedback differ such as adaptation of vertical growth to escape fires and lateral growth as herbivore defense \citep{archibald2003growing,staver2012top,moncrieff2011tree}. Recent study \citep{wigley2015mammal} suggests that, it is possible to add browse quality and defense type to the previously known trade-off in plant growth between herbivore versus fire-adapted woody species.

Plants can interact biochemically in a variety of ways, from stimulation to inhibition, that are ecologically significant\citep{muller1966role}. The effect of volatile compounds in enhanced flammability is well-known \citep{mutch1970wildland,white1994monoterpenes,owens1998seasonal,volatile1,volatile2,volatile3,alam2020shoot,ormeno2009relationship} and the ability of plant species to create and store certain types of chemical compounds might have adaptive value in fire-prone ecosystem \citep{pausas2016secondary}. However, the evidence for the primary function of the secondary compounds being defense is widespread \citep{primaryfunction} and plants-to-plants biochemical interaction can form a distinct type of ecosystem such as the shrub-fire-herb cycle in the California chaparral \citep{allelopathic}. A recent study suggests that, even in plants with little fire history, flammability can be emerged \citep{cui2020shoot}. Moreover, some volatile compounds act as herbivore defense as well as can enhance flammability \citep{white1994monoterpenes} and \citep{owens1998seasonal} suggested that, in some cases, the development of enhanced flammability might be the response to herbivores.

A plant can have physical (e.g., thorns, spines, or prickles), chemical (secondary compounds), or both types of defense against browsing animals. Studies tend to support the notion that defense is expensive and that there is a real trade-off between chemical and physical deterrents. \citep{rhoades1979evolution, van1988defence,twigg1996physicalchemical}. However, this view is not universal and sometimes inconsistent with other studies \citep{iddles2003potentialnegativecorrelation,steward1988theredifferentview,koricheva2004metanegativecorrelation}. Moreover, the plant defense can occasionally promote the development of a mutualistic relationship with other organisms \citep{janzen1966coevolution}, and herbivores can occasionally be advantageous to plants \citep{belsky1986does}. A recent study with a global data set showed that there is no conclusive evidence to support the trade-off and suggested that several defense characteristic combinations can be found in plants \citep{moles2013correlations}. Therefore, testing the flammability difference between armed (plants having e.g., spines, thorns, prickles) and unarmed plant species might improve the understanding of the trade-off.

Texas has a long history of fire \citep{moir1982firehistory, stambaugh2011firehistory,stambaugh2014historicalfirehistory,smeins2005historyoffire1} and grazing \citep{buechner1950lifegrazing, wilcox2012historicalgrazing2}. Studies suggest that, in many cases, the transformation of many open grasslands and savannas into woodlands and the dominance of many fire-sensitive and least preferred shrubs for herbivores in this region is due to the infrequent natural wildfires and overgrazing \citep{archer1989havejoint,andruk2014joint, masters1986prescribed}. Therefore, Texas is an ideal place to test the prediction related to life history strategies shaped by fire and herbivores as a consumer of plants. The preference of herbivores is a relative concept that depends on a number of variables, including soil type, plant's age, the degree of use, and the availability of alternative plants \citep{wright2003white}.  A recent study suggests that there is a negative correlation between flammability and digestibility in plants \citep{gowda2022digestibility}. Moreover, \citep{archibald2019unified} suggested that it is feasible to determine the extent to which fire and herbivore adaptations are hostile or linked once they are placed on common axes. Therefore, testing the difference in flammability between different groups of shrubs based on herbivores' preferences might improve the understanding of avoidance–attraction traits for fire and herbivore \citep{schwilk2003flammability, archibald2019unified}. 

In light of the recent development of the understanding of flammability, plant defense against herbivores, and browsing preference, I asked two questions. 1) Does least preferred plants are more flammable than moderate to low preferred plants?  2) Do armed plants have lower flammability than unarmed plants? This research could contribute to a better understanding of the ``Unified framework of the life history strategies of plants shaped by fire and browsing animals'' proposed by \citep{archibald2019unified}.


\section{Data}

In the summer of 2021, I collected a single 70\,cm healthy-looking terminal branch from individual shrubs, and I measured only flammability traits.  I have flammability metrics data from both years and there were 220 samples in total, representing 21 different shrub species. Only those samples that were ignited within the 10 seconds ignition period were used from 2021, since, in 2022, ignitibility was measured as ignition delay. The list of shrub species based on the preference of herbivores is collected from \citep{wright2003white} where they categorized the woody plants, vines, and cacti based on the preference for leaves and stems for white-tailed deer into four groups: High, Moderate, Low and Least Used. The species from the high preferred group is normally rare since they are heavily browsed in most of the cases and due to the limited number of species studied, I treated the high, moderate, and low preferred groups as a single group and assigned those species as moderate to low preferred. There were 171 samples in total, representing 16 different shrub species.\\

 \begin{figure}
     \centering
     \includegraphics[width = \textwidth]{../../results/map.pdf}
    
     \caption{Map showing the locations where samples were collected in 2022 and 2021,
     the colored dots are collection properties}
 \end{figure}

\begin{table}
    \centering
    \begin{tabular}{lrr}
          Unarmed &  Armed \\
          \hline
          \emph{Juniperus ashei} & \emph{Mahonia trifoliolata}\\
          \emph{Juniperus pinchotii} & \emph{Prosopis glandulosa}\\
          \emph{Juniperus virginiana} & \emph{Senegalia wrightii}\\
          \emph{Sophora secundiflora} & \emph{Senegalia berlandieri}\\
          \emph{Rhus virens} & \emph{Condalia hookeri}\\
          \emph{Diospyros texana} & \emph{Sarcomphalus obtusifoloa}\\
          \emph{Rhus trilobata} & \emph{Zanthoxylum fagara}\\
          \emph{Rhus microphylla} & \emph{Coleogyne ramosissima}\\
          \emph{Forestiera pubescens} & \emph{Mimosa borealis}\\
          \emph{Quercus virginiana} &   \\
          \emph{Ilex vomitoria} &    \\
          \emph{Calicarpa americana} & \\
    \end{tabular}
    \caption{List of armed and unarmed species in this study}
\end{table} 

\begin{table}
    \centering
    \begin{tabular}{lrr}
     Moderate to Low & Least \\
     \hline
    \emph{Rhus virens} & \emph{Juniperus ashei} \\
    \emph{Rhus trilobata} & \emph{Juniperus pinchotii}\\
    \emph{Rhus microphylla} & \emph{Juniperus virginiana}\\
    \emph{Quercus virginiana} & \emph{Prosopis glandulosa}\\
    \emph{Forestiera pubescens} & \emph{Diospyros texana}\\
    \emph{Mimosa borealis} & \emph{Sophora secundiflora}\\
    \emph{Zanthoxylum fagara} &   \\
    \emph{Senegalia wrightii} & \\
    \emph{Coleogyne ramosissima} & \\
    \emph{Senegalia berlandieri} & \\
    \end{tabular}
    \caption{List of shrubs for White-tailed deer preference \citep*{wright2003white}}
\end{table}

The herbivore’s preference for certain plant species is relative \citep{wright2003white}. The list of woody plants based on the white-tailed deer preference from \citep{wright2003white} is almost consistent with \citep{nelle1996management} who categorized the woody plants into four groups based on browsers’ preference from the Edwards Plateau in Texas. However,  \citep{wright2003white} put \emph{Diospyros texana} and \emph{Senegalia berlandieri} in both the low and least preferred groups while \citep{nelle1996management} categorized the \emph{Diospyros texana} as the least used and \citep*{varner1987southern} suggested that \emph{Senegalia berlandieri} contribute a significant portion of the diet of white-tailed deer in summer in this region.  Therefore, I decided to treat the \emph{Diospyros texana} as the least used and \emph{Senegalia berlandieri} as the moderate to low used group. Some other lists of plants from this study based on herbivores’ preferences in Texas can also be found \citep{arnold1979seasonallist, nelle2001ecological, everitt1974springfoodhabit, dillard2006whitetaileddeer}. Since I tested the difference in flammability between the unarmed and armed species, all the shrubs in the least preferred group are unarmed except \emph{Prosopis glandulosa} since the secondary chemicals are more effective as deterrents against herbivores than its physical defense for Honey mesquite \citep{wright2003white}.

\section{Analysis}
The temperature integration over 100 $^{\circ}$C is highly correlated with all the flammability metrics related to heat release for both 2021 and 2022. Therefore, temperature integration over 100 $^{\circ}$C as a proxy measurement of heat release was selected as the response variable. Initially, to test the significance of the difference in flammability between armed and unarmed species as well as between least preferred and moderate to low preferred species for white-tailed deer was analyzed using liner mixed-models by \pkg{lme} function from \pkg{nlme} package \citep{pinheiro2017package}. Due to the different number of samples for different groups, I  built two separate models, one with two groups of shrubs' based on defense strategy and another one with white-tailed deer preference as fixed term and the species as a random effect to account for the non-independence of samples due to repeated sampling over the same species. The response variable is not normally distributed. However, the resilience of the F-test is more influenced by heterogeneity than by non-normality \citep{blanca2017non}. Therefore, to account the variance structure between groups, the fixed factors weighted and \pkg{varIdent} function was specified to count the difference of variance in groups. All the models were fitted with Restricted Maximum Likelihood. 
% Leaving the post-hoc test part
%In order to test the significance of the difference of mean between different levels of fixed terms, I performed post-hoc test by \pkg{glht} function from \pkg{multcomp} package \citep{hothorn2016package}. The Tukey test was specified inside the multiple comparison procedure, \pkg{mcp} function in
%\pkg{glht} function.

%[Tukey test on two groups? That makes no sense. Why are you doing opst-hoc tests at all?]

\section{Results}

The linear mixed effect model with herbivore defense as fixed term showed that there is no statistically significant (F-value = 2.79, p-value = 0.111) difference in flammability between unarmed and armed species. %However, Tukey contrasts Post-hoc Test for multiple comparisons showed that the mean value of flammability of unarmed species is marginally higher than the armed species (p.adj = 0.094, 95\% C.I = [-1876.76, 23599.91]).
The linear mixed effect model with herbivore preference as fixed effect showed that there is a significant difference in flammability between least and moderate to low preferred shrub species (F-value = 9.08, p-value = 0.006)
(Figure \ref{fig:herbivore_pref}). 

%Tukey contrasts Post-hoc Test for multiple comparisons found that the mean value of flammability of moderate to low  preferred species is significantly lower than the least preferred species for White-tailed deer (p.adj = 0.002, 95\% C.I = [-36854.25, -7811.29])

%[DWS: But you can;t do the post-hoc test if the main model showed no effect. Remember your stats classes.]


\begin{figure}
     \centering
     \includegraphics[width= \textwidth]{../../results/herbivore_preference.pdf}
     \caption{Temperature integration ($^{\circ}$C.s) (x-axis) between least (red) and moderate to low (Blue) preferred species for White-tailed deer. The horizontal line inside the boxplot represents the median.}
     \label{fig:herbivore_pref}   
 \end{figure}

All the collected species were not available in every collection properties during sampling, and considering the difference in flammability that could arise due to the difference of collection properties for the unmeasured variables, I filtered the species which were found in at least two different collection properties. Then, I tested the difference in flammability among properties for each species and found that flammability in \emph{Prosopis glandulosa} marginally varies among properties (p = 0.09) and \emph{Sophora secundiflora}, \emph{Juniperus virginiana} and \emph{Forestiera pubescens} significantly varies in their flammability among properties (\emph{Sophora secundiflora}: p = 0.027, \emph{Juniperus virginiana}: p = 0.002, \emph{Forestiera pubescens}: p = 0.001). However, after removing those four species from the analysis, the relationships held true for the difference in flammability between the least and moderate to low preferred species (F = 13.20, p = 0.005).

\section{Discussion}

The result demonstrated that the least preferred species are more flammable than moderate to low preferred species for white-tailed deer. However, the terminologies for discussing fire and herbivory functional traits are not standardized \citep{archibald2019unified}. Therefore, the higher flammability of the least preferred group compared to the moderate to low preferred group supports the prediction by \citep{archibald2019unified}, who suggested that flammability and palatability-related traits are dissimilar from one another and a plant is more likely to burn in a fire if its life history strategy avoids being defoliated by animals. If we consider herbivore preference to be synonymous with palatability, which means ``Having leaf material that is preferred by grazers" \citep{archibald2019unified} then the higher flammability of the species from the least preferred group compared to the moderate to low preferred group supports that claim.

This study found that there is no statistically significant difference in flammability between unarmed and armed species. %However, the mean flammability of unarmed species is marginally higher than armed species. 
This observation supports the findings from other studies which suggested that the trade-off between different plant defenses is not robust \citep{steward1988there, koricheva2004meta, moles2013correlations}. Moreover, the role of chemical compounds as a defense against insect herbivores' \citep{herms1992dilemma, ohgushi2005indirect} is more pronounced than mammalian herbivores' across literature and studies suggested that the ability of chemical deterrents to stop consuming preferred plant materials for mammalian herbivores is limited \citep{cooper1985condensed, cooper1988foliage}. Therefore, though there might be a significant trade-off between different defense strategies against different kinds of herbivores' in different spatiotemporal scale in plants \citep{eck2001trade, wigley2015mammal, dostalek2016trade}, it is likely that, in mature shrubs, for mammalian herbivores, the trade-off between physical and chemical defense is not strong. Another possibility is that the number of grazing animals per unit of land has decreased drastically in this region in the recent past \citep{wilcox2012historicalgrazing2} and study suggests that the trade-off between plant defenses is dominated by evolutionary constraints \citep{eichenberg2015trade}. Therefore, it is likely that there is a threshold after which the trade-off is more evident, and the studied species didn’t cross that threshold due to the recent decrease in stocking densities in this region.

\backmatter

\bibliographystyle{ecology}
\bibliography{myref}

\end{document}